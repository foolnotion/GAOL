%// TODO: Document interval2f class
%// TODO: Document "make check" and "make perf"
%// TODO: Document static constants changed as functions
%// TODO: Add examples of u²se to all functionalities in manual
%// TODO: Document intervalf class
\documentclass{manual}

\usepackage{amsmath}
\usepackage{amssymb}
\usepackage{graphicx}
\usepackage{pifont}
\usepackage{multicol}


\title{Gaol \version}
\subtitle{NOT Just Another \\Interval Arithmetic Library}
\author{Fr\'ed\'eric Goualard}
\institute{Laboratoire d'Informatique de Nantes-Atlantique, France}

\hypersetup{pdftitle={Gaol: NOT Just Another Interval Arithmetic Library},pdfauthor={Frederic Goualard},pdfkeywords={constraint,interval arithmetic}}

\newcommand{\itv}[2]{\ensuremath{[#1,\,#2]}}
\newcommand{\roundDn}[1]{\ensuremath{\downarrow#1\downarrow}}
\newcommand{\roundUp}[1]{\ensuremath{\uparrow#1\uparrow}}
\newcommand{\roundNearest}[1]{\ensuremath{\updownarrow#1\updownarrow}}
\newcommand{\hull}[1]{\ensuremath{\mathop{\Box} #1}}
\newcommand{\leftBound}[1]{\ensuremath{\underline{#1}}}
\newcommand{\rightBound}[1]{\ensuremath{\overline{#1}}}
\newcommand{\returns}{\hbox{\ding{229}}}
\let\leq=\leqslant
\let\geq=\geqslant
\let\emptyset=\varnothing

%% Non terminal
\newcommand*{\NT}[1]{\uppercase{\textit{#1}}}
%% Terminal
\newcommand*{\TM}[1]{\textsf{#1}}


\newcommand*{\prevfloat}[1]{\ensuremath{{#1}^-}}
\newcommand*{\nextfloat}[1]{\ensuremath{{#1}^+}}

\input{gaol_version}

\begin{document}
\maketitle

\begin{copyrighttxt}
Copyright \copyright{} 2001 Swiss Federal Institute of Technology, Switzerland

Copyright \copyright{} 2002-2016 LINA UMR CNRS 6241, France

Copyright \copyright{} 2017-\number\year\ LS2N UMR CNRS 6004, France

All Trademarks, Copyrights and Trade Names are the property of their respective owners
even if they are not specified below.

Part of the work was done while Fr\'ed\'eric Goualard was a postdoctorate
at the \emph{Swiss Federal Institute of Technology, Lausanne, Switzerland}
supported by the \emph{European Research Consortium for Informatics and
Mathematics} fellowship programme.

This is edition \edition{} of the gaol documentation.
It is consistent with version \version{} of the \texttt{gaol} library.

Permission is granted to make and distribute verbatim copies of this manual
provided the copyright notice and this permission notice are preserved on all
copies.

\smallskip
GAOL IS PROVIDED "AS IS" WITHOUT WARRANTY OF ANY
KIND, EITHER EXPRESSED OR IMPLIED, INCLUDING, BUT NOT LIMITED TO, THE
IMPLIED WARRANTIES OF MERCHAN\-TA\-BILITY AND FITNESS FOR A PARTICULAR
PURPOSE.  THE ENTIRE RISK AS TO THE QUALITY AND PERFORMANCE OF THIS
SOFTWARE IS WITH YOU.  SHOULD THIS SOFTWARE PRO\-VE DEFECTIVE, YOU ASSUME
THE COST OF ALL NECESSARY SERVICING, REPAIR OR CORRECTION.
\end{copyrighttxt}

\newpage
\thispagestyle{empty}

\vspace*{7cm}
\marginpar{\begin{raggedright}
  \textbf{Recipriversexcluson. n.}
  \textit{A number whose existence can only be defined as being anything other than itself.} \par
\begin{flushright}
  Douglas Adams, {\fontfamily{cmr}\selectfont\oldstylenums{1982}}\\
  \textit{Life, the Universe and Everything}
\end{flushright}
\end{raggedright}}

\cleardoublepage

\tableofcontents

\chapter*{Copyright}\addcontentsline{toc}{chapter}{Copyright}
%%=================
Gaol is distributed under the GNU Lesser General Public License (\pxref{chap:copying}).
The copyright for the initial version---named \code{cell}---(year 2001) is owned by the
\findex{cell}%
\emph{Swiss Federal Institute of Technology}, Lausanne, Switzerland.
The copyright for the following versions (from 2002 onward) is owned by the
\emph{Laboratoire d'Informatique de Nantes-Atlantique}, France.

\medskip
For the input of floating-point numbers, gaol uses the \code{strtord()} function written
by David M. Gay, whose copyright notice follows:
\findex{strtord!extern}%

\begin{verbatim}
 * The author of this software is David M. Gay.
 *
 * Copyright (c) 1991, 2000, 2001 by Lucent Technologies.
 *
 * Permission to use, copy, modify, and distribute this
 * software for any purpose  without fee is hereby  granted,
 * provided that  this entire notice is included in all
 * copies of any software  which is or includes  a copy or
 * modification  of this software and  in all copies of the
 * supporting documentation for such software.
 *
 * THIS SOFTWARE IS BEING PROVIDED "AS IS", WITHOUT ANY
 * EXPRESS OR IMPLIED WARRANTY.  IN PARTICULAR, NEITHER
 * THE AUTHOR NOR LUCENT MAKES ANY REPRESENTATION OR
 * WARRANTY OF ANY KIND CONCERNING THE MERCHANTABILITY
 * OF THIS SOFTWARE OR ITS FITNESS  FOR ANY PARTICULAR
 * PURPOSE.
\end{verbatim}


Gaol relies either on IBM APMathlib or CRlibm mathematical libraries for most floating-point
operators. Both libraries are released under the GNU Lesser General
Public License (\pxref{chap:copying}).
\cindex{APMathlib}\cindex{CRlibm}%

\mainmatter


\chapter{Introduction}
%%====================


Gaol$^\dag$\marginpar{$^\dag\:$For those readers who are not native
  English speakers, ``gaol'' should be pronounced \textsf{j\={a}l},
  like the word ``jail'', of which it is a, chiefly British, variant.}
\cindex{gaol!pronunciation}%
is a C++ library to perform arithmetic with floating-point intervals.
The development of gaol was initiated at the \emph{Swiss Federal
  Institute of Technology}, Lausanne, Switzerland, while F.  Goualard
was a post-doctorate fellow supported by the \emph{Swiss National Science
  Foundation}.  It started as a limited version of \emph{Jail} (now
\emph{halloween}), a templated C++ interval library developed during
Goualard's PhD.

To our knowledge, a unique feature of gaol among all C++ interval
arithmetic libraries available is the implementation of
\emph{relational arithmetic operators} required by interval constraint
arithmetic software (\pxref{sec:relational-arithmetic}).
\cindex{operator!relational}\cindex{relational operator}%
Hence, the game of the name: gaol is \emph{not} JAIL (\emph{Just
  Another Interval Library}). That situation should change in the near future, as 
relational arithmetic operators are required by the future IEEE 1788 standard for
interval arithmetic.

This document is both a manual and a reference to use gaol. It assumes
a prior knowledge of interval arithmetic. Refer to the books and
papers by Goldberg, Neumaier, and
others~\margincite{Goldberg:01,IEEE754:90,Moore:Book1966,Alefeld-Herzberger:83,Neumaier:Book1990}
for a basic presentation of floating-point arithmetic, interval
arithmetic and the use thereof.
\cindex{references!on interval arithmetic}

The main entry point for interval arithmetic on the Web is Vladik
Kreino\-vich's \emph{Interval Computation site} (\url{http://www.cs.utep.edu/interval-comp/}).
\cindex{interval!computation site (web)}

Classes, methods, functions, macros, constants and variables available in the
library but not described in this document are likely to change or to
be removed. Consequently, they should be used with caution, if at all.
\cindex{undocumented!use of \idxself{} functions}%
\cindex{function!undocumented}%

\chapter{Installation}
%%====================

The installation procedure differs depending on your platform. The current release of
gaol is supported on the following platforms:

\begin{itemize}
\item ix86-based computers and compatibles under Linux with 
  GNU gcc/g++. Both 32 bits and 64 bits operating systems are supported;
\item ix86\_64-based computers under Mac OS X (gaol has been successfully tested on 
\textit{Darwin}).
\end{itemize}

\noindent Gaol used to be available on UltraSparc-based computers
under SUN Solaris 2.[5--8] with GNU gcc/g++. With no such architecture
at hand anymore, it is no longer actively developed on it, though
its support should not require too much work.
\cindex{sparc!installing gaol on}\cindex{solaris!installing gaol on}%

\section{Getting the software}
%%----------------------------

The official web page for gaol is \url{http://sourceforge.net/projects/gaol/}.
\cindex{gaol!web page}%

The latest versions of gaol come as a source-code tarball only, which should be compiled 
and installed specifically for the computer you intend to use. There are no precompiled 
versions.


\section{Installing gaol from the source tarball on Unix and Linux}
\label{sec:installing-gaol-unix}
%%-----------------------------------------------------------------

Installing gaol from the source archive is done in three steps,
in accordance with the spirit of all GNU softwares: \emph{configuration,
building, installation}. These steps are described hereunder. In the
following, the base directory of the gaol distribution as created by
decompressing the archive will be referred as the \dfn{root directory
of the distribution} (or simply, the root directory).

\subsection{Prerequisites}
%%------------------------
In order to build and install gaol, you will need the following
tools. Some of them are mandatory, some are only required if you
intend to modify the code, and others are only optional (their
absence will not prevent you from using gaol though some features might be
unavailable).

\subsubsection{Mandatory tools and programs}
%%..........................................

Gaol uses many features provided by the ANSI-standard ISO/IEC FDIS 14882
for the C++ language. As a consequence, you will need a recent C++
compiler in order to compile gaol---e.g. \href{http://gcc.gnu.org}{gcc 3.0} or above.
\index[idx]{c++!standard}%
\cindex{tool!mandatory \idxself{} to compile gaol}%

Gaol relies either on IBM APMathlib or CRlibm floating-point arithmetic libraries. They must be
properly installed on your system prior to configuring gaol. For your convenience,
an archive of APMathlib is available on the gaol web site (\url{http://sourceforge.net/projects/gaol}).
The CRlibm library is available on LIPForge (\url{http://lipforge.ens-lyon.fr/projects/crlibm/}).
\index[idx]{APMathlib}\index[idx]{CRlibm}%

\paragraph{Important note.} Some assumptions are made concerning the accuracy of some functions provided by the standard
mathematical library (\code{libm}). In particular, depending on the library chosen (\code{apmathlib} or \code{crlibm}),
the following functions are considered to be computed to at most one ulp of accuracy:
\cindex{accuracy!of floating-point functions}\cindex{requirements!accuracy}
\begin{center}
 \begin{tabular}{l|cc}
	\hline
	Functions & \textbf{APmathlib} & \textbf{CRlibm}\\
	\hline
	cosh      &  $\bullet$ & \\
	sinh      &  $\bullet$ & \\
	tanh      &  $\bullet$ & $\bullet$\\
	acosh     &  $\bullet$ & $\bullet$\\
	asinh     &  $\bullet$ & $\bullet$\\
	atanh     &  $\bullet$ & $\bullet$\\
	\hline
 \end{tabular}
\end{center}


\subsubsection{Tools for maintainers}
%%...................................

Gaol uses code produced by GNU Flex and GNU Bison for parsing the expression used to
initialize an interval.
\findex{flex}%
\findex{bison}%

Any modification of the files
\begin{center}
\file{gaol\_interval\_lexer.lpp}
\end{center}
\noindent or
\begin{center}
\file{gaol\_interval\_parser.ypp}
\end{center}
requires the availability of these tools.

\subsubsection{Optional tools}
%%............................

\begin{itemize}
\item \cmd{dot}. This program is used by \cmd{doxygen}
\findex{dot|see{GraphViz}}%
(see below) to draw dependency graphs in the HTML documentation. It is
part of the GraphViz package (\url{http://www.research.att.com/sw/tools/graphviz/});
\findex{Graphviz}%
\item \cmd{doxygen}. Tool similar to SUN Javadoc for the C++ language. It is
available at \url{http://www.stack.nl/~dimitri/doxygen/index.html}.
If you do not have it, you will not be able to regenerate the HTML documentation;
\index[idx]{doxygen}%
\item \cmd{CppUnit}. This library for unit testing is available on SourceForge
  (\url{http://sourceforge.net/projects/cppunit}). It is required to test the
  proper compilation of gaol.
\index[idx]{CppUnit}%
\end{itemize}

\subsection{Configuration}\label{sec:configuration}
%%------------------------

Before actually compiling the library, you have to configure it for your
platform by using the \cmd{configure} program
located at the root of the gaol distribution. It accepts the following options:
\findex{configure}%

\begin{itemize}
\item \loption{help}. Displays a list of all options. Note that only
those described hereunder are supported;
\cindex{help!on configuration}%
\item \loption{prefix=\var{prefix-dir}}. The root directory where the library
will be installed. It defaults to \file{/usr/local};
\item \loption{libdir=\var{lib-dir}}. The directory where to put the
libraries. It defaults to \file{\var{prefix-dir}/lib};
\item \loption{includedir=\var{include-dir}}. The directory where to
put the header files. It defaults to \file{\var{prefix-dir}/include};
\item \loption{infodir=\var{info-dir}}. The directory where to put the
documentation in \file{info} format. It defaults to \file{\var{prefix-dir}/info};
\item \loption{enable-shared[=\var{yes/no}]}. Creates or not a shared
library. This option defaults to \cmd{yes} whenever shared libraries
are supported by the current platform;
\item \loption{with-mathlib=\var{(apmathlib|crlibm)}}. Specifies which mathematical library to use.
  The following libraries are currently supported.
\begin{description}
 \item[\cmd{apmathlib}.] The IBM APMathlib library. This library must be installed and available before configuring \emph{gaol};
 \item[\cmd{crlibm}.] The \emph{Correcty Rounded Mathematical library} available at \url{http://lipforge.ens-lyon.fr/www/crlibm/}.
	This library should be installed and available before configuring \emph{gaol}.
\end{description}
  \cindex{mathematical library}\cindex{library!mathematical}
  This option defaults to ``\cmd{apmathlib}.''
\item \loption{with-mathlib-include=PATH}. Set the path where are the headers for the mathematical
  library to PATH;
\cindex{mathlib!include path}
\item \loption{with-mathlib-lib=PATH}. Set the path where is the library file for the mathematical
  library to PATH;
\cindex{mathlib!library path}
\item \loption{with-cppunit-include=PATH}. Set the path where are the headers for cppunit
  to PATH;
\cindex{cppunit!include path}
\item \loption{with-cppunit-lib=PATH}. Set the path where is the library file for cppunit to PATH;
\cindex{cppunit!library path}
\item \loption{enable-debug[=\var{yes/no}]}. Adds or not debugging
information to the library. Enable the use of debugging macros (\pxref{sec:debugging-facilities}).
This option defaults to \cmd{no}.
\cindex{debugging!option}%
\item \loption{enable-preserve-rounding[=\var{yes/no}]}. The library
  assumes that the rounding direction is never modified outside of
  gaol, which allows to set it once and for all to "upward" at
  initialization (\pxref{sec:trust-rounding-mode}).
  This option defaults to \cmd{no}. You should define this option to \cmd{yes} if you use
  gaol together with libraries or programs that manipulate the rounding direction, or that
  require the rounding direction to be to the nearest;
\cindex{rounding!preserving (\texttt{gcc})}%
\item \loption{enable-optimize[=\var{yes/no}]}. Compiles gaol with full optimization turned on.
  This option defaults to \code{yes};
\item \loption{enable-fast-math[=\var{yes/no}]}. Compiles gaol with fast but less accurate
  transcendental and power operators. This option defaults to \code{yes};
\cindex{optimization!configuration option}%
\item \loption{enable-exceptions[=\var{yes/no}]}. If enabled, errors
  should be reported by throwing an exception (\pxref{sec:exceptions}). If
  disabled, errors are reported by calling \code{gaol\_error()}, which
  prints a message to the standard error channel. This option defaults
  to \code{yes};
\cindex{error!raising an exception}%
\item \loption{enable-asm[=\var{yes/no}]}. Allows the use of assembler
  code in some parts of gaol. For most platforms, assembler code is
  used only to switch thr rounding direction of the FPU. On ix86,
  assembler code is also used for many primitives. This option
  defaults to \code{yes}. On ix86, disabling the assembler support
  should be done with caution, depending on the propensity of of your
  compiler to wrongly optimize code using floating-point instructions
  (which is high for versions of \code{gcc} prior to 4.1);
\cindex{assembler}%
\item \loption{enable-relations=\var{kind}}. Defines the kind of relation
  to use for relational symbols ($=,\leq, \dots$) to be
  \var{kind}. Possible values for \var{kind} are
  \code{certainly}, \code{set}, and \code{possibly} (\pxref{sec:interval-relations}).
  The default is \code{certainly};
\cindex{relation}%
\item \loption{verbose-mode[=\var{yes/no}]}. Allows information messages to be sent to the standard
  output (such as messages to report automatic initialization and cleanup).
  The default is \code{yes}.
\cindex{verbose mode}%
\cindex{messages!avoiding printing of}%
\item 
\end{itemize}

Note that, as usual, \loption{disable-\var{xxx}} is
equivalent to \loption{enable-\var{xxx}=no}. Moreover,
\loption{enable-\var{xxx}} is equivalent to
\loption{enable-\var{xxx}=yes}.

\subsubsection{Configuration examples}
%%....................................

First, go to the root directory. If you simply type
\begin{cmdshell}
% ./configure
\end{cmdshell}
you will create a shared library with full optimization, which will be
installed in \file{/usr/local}.
\cindex{library!shared}\cindex{shared|see{library}}

\noindent By issuing

\begin{cmdshell}
% ./configure --enable-debug \
              --prefix=/usr1/local --infodir=/export/info
\end{cmdshell}

you will create a library including
debugging information that will be installed in the root directory
\file{/usr1/local} except for the info files, which will be installed in
the \file{/export/info} directory.

Missing optional tools are reported during the configuration process,
though they do not prevent you from building the library. The
configuration is aborted if some important tool or library is missing.


\subsection{Building}
%%-------------------

After having configured gaol, you can now type

\begin{cmdshell}
% make
\end{cmdshell}
in the root directory to build the library and its documentation
(\cmd{pdf} and \cmd{html} files).

The targets for the \file{Makefile} in the root directory are:

\begin{itemize}
\item \option{all}. Similar to calling \cmd{make} without any
  argument;
\item \option{doc}. Create the manual in both \cmd{pdf} and
  \cmd{html} formats;
\item \option{html} Create only the \cmd{html} reference;
\item \option{check}. Test the library by compiling some benchmarks and
  checking their output against the expected one;
\item \option{clean}, \option{distclean},
  \option{maintainer-clean}. These are standard options for a GNU standard
  compliant \file{Makefile}. The \option{clean} option erases all files
  created during the building process; the \option{distclean} erases also
  the files created during the configuration process;
  \option{maintainer-clean} is meant to be used by maintainers only since
  it might erase files needing special tools to be re-created as well;
\item \option{install}. Install the library in the directories specified
  at configuration time.
\end{itemize}


\subsection{Installation}
%%-----------------------
To install gaol on your system, just type

\begin{cmdshell}
% make install
\end{cmdshell}

Remember that the directories you have chosen to install the libraries
into, must be accessible to your compiler, i.e. they must appear in
the paths contained in the relevant environment variables:
\begin{itemize}
\item \env{LIBRARY\_PATH} for static libraries,
\item \env{LD\_LIBRARY\_PATH} for dynamic libraries,
\item \ldots
\end{itemize}
to be able to use the library once installed.
\findex{LD\_LIBRARY\_PATH}%


\chapter{An overview of gaol}
%%===========================

 In this chapter, we will assume that gaol has already been
 properly installed, and that the libraries and header files are
 accessible to your compiler.

 Let us consider the following program to compute the
 range of the function
 \begin{align*}
 f(x,y)=&\bigl(1+(x+y)^2(19-14x+3x^2-14y+6xy+3y^2)\bigr)\\
 &\bigl(30+(2x-3y)^2(18-32x+12x^2+48y-36xy+27y^2)\bigr)
 \end{align*}
 for $x\in\itv{-2}{2}$ and $y\in \itv{-2}{2}$.

 \begin{example}
 #include <iostream>
 #include <gaol/gaol.h>

 int main(void)
 {
   gaol::init();

   interval
     x(-2,2),
     y(-2,2), z;

   z=(1+sqr(x+y)*(19-14*x+3*sqr(x)-14*y+6*x*y+3*sqr(y)))*
     (30+sqr(2*x-3*y)*(18-32*x+12*sqr(x)+
        48*y-36*x*y+27*sqr(y)));

   std::cout << "z = " << z << std::endl;

   gaol::cleanup();
   return 0;
 }
 \end{example}

 First, note that we have to include the \file{gaol/gaol.h} header file in order to
 use all the facilities provided by gaol. All the functions, classes,
 constants and types defined in gaol are embedded into the \code{gaol}
 namespace. The \code{gaol/gaol.h} header imports the whole namespace such
 that it is not necessary to use the \code{gaol} prefix. Alternatively,
 you may include the \file{gaol/gaol}
 header instead of \file{gaol/gaol.h} and add \code{using} directives to only
 import what you actually need.
\cindex{gaol!namespace}%
\cindex{namespace!gaol}%
\cindex{header!and namespace|see{namespace}}%

The call to \code{gaol::init()} on Line~6 is related to the use of the
so-called \dfn{trust rounding mode} (see next section): it switches
the rounding mode of the floating-point unit towards $+\infty$ (if the
library was compiled with \loption{preserve-rounding=\cmd{no}}) and
calls some initialization code.
\findex{init}\cindex{rounding!preserving}%

 The \code{sqr(\var{x})} function stands for \emph{square of \var{x}}
 and is equivalent to \code{pow(\var{x},2)}.

 Let \file{f.cpp} be the name of the file containing the program
 above. To compile it with \cmd{g++}, we have to type the following command:
\cindex{compiling!a file using gaol}%

 \begin{cmdshell}
 % g++ -o f f.cpp -lm -lultim -lgaol
 \end{cmdshell}
 were \code{ultim} corresponds to the APMathlib library.
\findex{ultim|see{APMathlib}}\findex{libultim|see{APMathlib}}%

 We thus create the executable file \file{f}, using the \code{gaol} and
 \code{APMathlib} libraries---\code{APMathlib} is the
 \href{http://oss.software.ibm.com/mathlib/}{\emph{Accurate Portable Mathematical library}}
 developed by IBM; it provides us with correctly rounded mathematical functions if they are not
 directly available on the given platform.
\index[idx]{APMathlib|hyperpagebf}%

 Executing \file{f}, we obtain:

 \begin{onscreen}
  z = @itv^-56254330~^94177270~
 \end{onscreen}

 We then know that $f(x,y)$ ranges over
 \itv{-56254330}{94177270} when $x$ and $y$ range over \itv{-2}{2} independently.

\section{The trust rounding mode}\label{sec:trust-rounding-mode}
%%-------------------------------

 Floating-point interval arithmetic requires \dfn{outward rounding} in
 order to fulfill the \dfn{containment property}: for example, to add
 intervals \itv{a}{b} and \itv{c}{d}, we compute
 \itv{\roundDn{a+c}}{\roundUp{b+d}}, where \roundDn{r}
 and \roundUp{r} return the greatest (resp. smallest)
 floating-point number smaller (resp. greater) than the real result of
 $r$. These two operations are performed by switching the rounding
 direction of the FPU towards, respectively, $-\infty$ and
 $+\infty$.
 \marginpar{%
   \begin{definition}[Rounding down/up]
     \cindex{rounding!downward|hyperpagebf}\cindex{rounding!upward|hyperpagebf}%
     \cindex{F@\FSet}\cindex{R@\RSet}%
     Given \RSet\ the set of real numbers and \FSet\ the set of
     floating-point numbers (\texttt{double}), we have:
     \begin{equation*}
       \forall x\in\RSet\colon
       \left\{\begin{array}{l}
           \roundDn{x} = \max\{y\in\FSet\mid y\leq x\}\\
           \roundUp{x} = \min\{y\in\FSet\mid y\geq x\}
         \end{array}\right.
     \end{equation*}
   \end{definition}
 }
\cindex{containement property}%
\cindex{rounding!outward|see{outward rounding}}%
\cindex{downward rounding ($\roundDn{\:}$)}%
\cindex{upward rounding ($\roundUp{\:}$)}%
\cindex{outward rounding}%
\index[idx]{aaa@$\roundDn{\:}$|see{downward rounding}}%
\index[idx]{aab@$\roundUp{\:}$|see{upward rounding}}%

On most platforms, switching the rounding direction is costly.
However, it is possible to cut down the number of switches
by relying on the property that
$\roundDn{-r}=-\roundUp{r}$. Consequently, one can
replace nearly all downward rounding operations by upward rounding ones
by negating appropriately twice the operations performed. The next step
is then to only switch once and for all the rounding direction towards
$+\infty{}$ at the beginning of a computation. This strategy
reduces drastically the number of rounding direction switches at the cost of
putting on the user the burden to ensure that the rounding direction
be always set towards $+\infty{}$ before any computation involving
intervals. This mode is called the \dfn{trust rounding mode} since
we trust the user for ensuring that the rounding direction is always
properly set.

Basically, all the user has to do when using the trust rounding mode
is to switch the rounding direction towards $+\infty{}$ at the beginning of
his program (this is performed automatically by \code{gaol::init()}),
and then ensuring that it always remains set to that direction
before performing any interval operation. This
condition is never violated by any function or method of the library
itself.

When gaol is used in a larger application that also relies on libraries that assume the rounding
\important%
direction to be to the nearest, it is easier and safer to configure it with the trust rounding
mode disabled (\pxref{sec:configuration}).

\section{Common errors}
%%---------------------

In this section, we will review common errors made when using gaol in
the---forlorn?---hope that it will help prevent users from making
them.

\subsection{Floating-point arithmetic and rounding}
%%.................................................

Programming with floating-point numbers is one of the few activities
where one must always consider ones compiler defiantly. For example, let
us consider the following piece of code:

\begin{example}
#include <gaol/gaol>  @rem^// We do not import the gaol namespace~

using gaol::interval;

int main(void) {
  gaol::init();
  interval one_tenth(0.1); @rem^BEWARE: wrong !~

  @emph^[some code using one_tenth]~
  gaol::cleanup();
}
\end{example}

\marginpar{%
\begin{definition}[Rounding to the nearest]
  \cindex{Rounding!to nearest|hyperpagebf}
     Given \RSet\ the set of real numbers and \FSet\ the set of
     floating-point numbers (\texttt{double}), we have:
       \begin{multline*}
         \forall x\in\RSet\colon \roundNearest{x} = y\in\FSet
         \text{ s.t. }\\
         x-y=\min\{|\delta|\in\RSet\mid \delta=x-z, \forall z\in\FSet\}
     \end{multline*}
\end{definition}
}
Though a rational perfectly representable in decimal, \code{0.1} is
not representable in binary (at least, not with a finite number of
bits) and thus requires rounding. Obviously, the purpose of the user
was to define an interval containing this value.  However, \code{0.1}
will be rounded \emph{at compile time}, most certainly to the nearest
representable floating-point number \roundNearest{0.1}.  As a
consequence, \code{one\_tenth} will be a degenerate interval
containing only \roundNearest{0.1}, and the \dfn{containment property}
will be violated.

The right way to deal with rational constants that might not be perfectly
representable as floating-point numbers is to stringigy them, such
that they can be correctly rounded downward and upward at \emph{runtime}:

\begin{example}
#include <gaol/gaol>

using gaol::interval;

int main(void)
{
  gaol::init();
  interval one_tenth("0.1"), @rem^// OK: this is the right way~
           one_tenth2("1/10"); @rem^// Another possible way~

  @emph^[some code using one_tenth]~
  gaol::cleanup();
  return 0;
}
\end{example}

\marginpar{%
  \begin{definition}[canonical interval]
    \cindex{interval!canonical|see{canonical interval}}%
    \cindex{canonical interval|hyperpagebf}%
    A non-empty interval $I=\itv{a}{b}$ is \emph{canonical} if and only if
    $a\geq\prevfloat{b}$.
  \end{definition}
}
Now, \code{one\_tenth} will be the smallest floating-point interval
enclosing \code{0.1}. An interval like this one, containing at most two
consecutive floating-point numbers, is called a \dfn{canonical interval}.
\cindex{canonical interval}%

\chapter{Initialization and cleanup}
%%==================================

The following functions have to be called before using any functionality of
the library and just after having used it for the last time.

Since Release 1.0 of gaol, there is an automatic
initialization/cleanup feature that ensures that no problem will
arise if the user forgets to explicitly call these functions.

\begin{deffun}{bool}{init}{(int \var{dbg\_lvl} = 0)}
Initializes the variable \code{debug\_level} (\pxref{sec:debugging-facilities}) to the
value of \var{dbg\_lvl}.

If the library was not compiled with the
\option{-{}-enable-preserve-rounding} option, it sets the FPU control
word according to the requirements of the \code{APMathlib} library and
enforces rounding towards $+\infty$.
In addition, it sets the number of digits to display for interval bounds to 16.

Returns \code{true} if the library was not already initialized and false otherwise. After its first
call, the only effect of this function is to---possibly---set the debugging level to a new value.
\end{deffun}

\begin{deffun}{bool}{cleanup}{(void)}
Restores the state of the FPU to its value prior to the initialization of the
gaol library.

In the current version, returns \code{true} the first time it is called and false afterwards.
\end{deffun}


\begin{example}
#include <gaol/gaol>

int main(void)
{
  init(1); @rem^// First level of debugging requested~

  @emph^[Some code using interval arithmetic]~

  cleanup();
  return 0;
}
\end{example}

\chapter{Interval creation and assignment}
%%========================================

The methods for creating an interval and assigning a new value to an already existing one
are described in the following.

\section{Constructors}
%%--------------------

One can create an interval in five different ways:

\begin{itemize}
\item by providing its left and right bounds:
  \begin{example}
    interval x(@var^l~,@var^r~);
  \end{example}
  where \var{l} and \var{r} are \code{double}s or of a type that is
  castable into a \code{double};

\item by providing only one bound, for degenerate point intervals:
  \begin{example}
    interval x(@var^v~);
  \end{example}
  This is equivalent to: \code{interval x(\var{v},\var{v});}
\item without providing any bound:
  \begin{example}
    interval x;
  \end{example}
  This is equivalent to:
  \begin{example}
    interval x(-GAOL_INFINITY,+GAOL_INFINITY);
  \end{example}
  \noindent where \code{GAOL\_INFINITY} represents the infinity value of the \code{double} format
  (\pxref{sec:floating-point-constants});
\item by copying an already existing interval (\emph{copy constructor}):
  \begin{example}
    interval x(-12,12), y=x, z(x);
  \end{example}
\item by using a string representing an interval in the same format as
  the one used for input (\pxref{sec:input-format})

  \begin{example}
    interval x("[-23, inf]"),
             y("[5*0.1+dmin, 89*sinh(2.1)]");
  \end{example}

  If the input string does not comply with the expected format,
  an empty interval is returned. The exception \code{gaol::input\_format\_error} is raised
  if the library was compiled with exception support;
  in the absence of exception support, the \code{gaol\_error()} function is
  called to print an error message, and the \code{errno} variable is set to \code{-1}
  (\code{errno} is \emph{not} modified when no error occurs).
\item by using a string representing an interval in the same format as the one used for input
  (\pxref{sec:input-format}) for each bound:

  \begin{example}
 @rem^// constructs x=[-4,2]~
 interval x("[-5,4]+1","[4,6]-[3,2]");
  \end{example}
\end{itemize}

\paragraph{Caution.} you have to be very careful when creating an interval
from floating-point constants. Remember that a rational number that is
perfectly representable in the decimal base may require rounding in the
binary base. For example, if you write the following statement:

\begin{example}
  interval x(0.1);
\end{example}

\noindent
you will \emph{not} have created an interval containing $0.1$ since
this number has an infinite expansion in the binary base (i.e. it is
impossible to represent it perfectly whatever the size of the mantissa
may be). As a consequence, the constant $0.1$ has very likely
been rounded to the nearest floating-point number at compile-time.
In such a case, you have to use a string instead:

\begin{example}
interval x("0.1");
\end{example}

\paragraph{Note.}\label{par:inf_inf_note}
For performance reasons, interval constructors do not forbid to create intervals whose bounds are infinities of
the same sign:
\begin{example}
interval I(GAOL_INFINITY,GAOL_INFINITY);
cout << I << endl;
@outputs^<inf, inf>~
\end{example}
However, such intervals are neither allowed nor supported in interval computation. They are not considered empty intervals and their use with interval operators leads to inconsistent results.
\cindex{interval!disallowed}\cindex{interval!with infinite bounds!forbidden}%

As an exception, creating an interval from the \element{inf} string constant leads to a legitimate interval:
\begin{example}
interval I("[-inf, -inf]");
interval J("<-inf, -inf>");
interval K("inf");
cout << I << endl;
cout << J << endl;
cout << K << endl;
@outputs^[-inf, -1.797693134862316e+308]~
@outputs^[-inf, -1.797693134862316e+308]~
@outputs^[1.797693134862316e+308, inf]~
\end{example}

\section{Straight assignment}
%%---------------------------

It is possible to assign a new value to an already existing interval in
three different ways:

\begin{itemize}
\item by copying another interval:
  \begin{example}
    interval x(-12,12),
             y; @rem^// Here, y is @itv^-@infty^~+@infty~~
    y = x; @rem^// Now, y is @itv^-12~^12~~
  \end{example}
\item by using a string whose format follows the one expected for
input (\pxref{sec:reading-intervals}).

\begin{example}
interval x;
x = "[-inf, 123]";
\end{example}

\item by using a \code{double}:

\begin{example}
interval x;
x = 1234.5; @rem^// Now, x is @itv^1234.5~^1234.5~~
\end{example}
\end{itemize}

Note that there is no method for modifying a bound of an
interval since intervals must be considered as an atomic concept.

\section{Assignment combined with an operation}
%%=============================================

The following assignment operators combine the value of the interval
pointed to by \code{self} and the value of the right-hand side
interval.
\marginpar{%
    In this manual, we note \code{self} the object to which
    \code{this} is a pointer in C++.
\cindex{this@\code{this}}\cindex{self@\code{self}}}

\begin{defmethod}{interval}{interval\&}{operator\&=}{(const interval\& \var{I})}
  \begin{operation}
    \op{$\code{self}\gets\code{self} \cap I$}
  \end{operation}
Assigns to \code{self} the interval resulting from the intersection of
\code{self} and \var{I}.

\begin{example}
interval x(-12,12);
x &= interval(-6,23); @rem^// Now, @code^x~ is @itv^-6~^12~~
\end{example}

\end{defmethod}

\begin{defmethod}{interval}{interval\&}{operator|=}{(const interval\& \var{I})}
  \begin{operation}
    \op{$\code{self}\gets\code{self} \cup I$}
  \end{operation}
  Assigns to \code{self} the interval resulting from the union of \code{self} and \var{I}.

\begin{example}
  interval x(-12,12);
  x |= interval(-6,23); @rem^// Now, @code^x~ is @itv^-12~^23~~
\end{example}

\end{defmethod}

\defmethodx{interval}{interval\&}{operator+=}{(const interval\& \var{I})}
\begin{defmethod}{interval}{interval\&}{operator+=}{(double \var{d})}
  \begin{operation}
    \op{$\code{self} \gets \code{self} + I$}
    \op{$\code{self} \gets \code{self} + d$}
  \end{operation}

  Assigns to \code{self} the interval resulting from adding \code{self} and \var{I}
  (resp. \var{d}).

\begin{example}
interval x(-12,12);
x += interval(-6,23); @rem^// Now, @code^x~ is @itv^-18~^35~~
\end{example}
\end{defmethod}

\defmethodx{interval}{interval\&}{operator-=}{(const interval\& \var{I})}
\begin{defmethod}{interval}{interval\&}{operator-=}{(double \var{d})}
  \begin{operation}
    \op{$\code{self}\gets \code{self} - \var{I}$}
    \op{$\code{self}\gets \code{self} - \var{d}$}
  \end{operation}
  Assigns to \code{self} the interval resulting from subtracting \var{I} (resp. \var{d})
  from \code{self}.
\begin{example}
interval x(-12,12);
x -= interval(-6,23); @rem^// Now, @code^x~ is @itv^-35~^18~~
\end{example}
\end{defmethod}

\defmethodx{interval}{interval\&}{operator*=}{(const interval\& \var{I})}
\begin{defmethod}{interval}{interval\&}{operator*=}{(double \var{d})}
\begin{operation}
\op{$\code{self}\gets \code{self} \times\var{I}$}
\op{$\code{self}\gets \code{self} \times\var{d}$}
\end{operation}
Assigns to \code{self} the interval resulting from multiplying \code{self} and \var{I}
(resp. \var{d}).
\begin{example}
interval x(-12,12);
x *= interval(-6,23); @rem^// Now, @code^x~ is @itv^-276~^276~~
\end{example}
\end{defmethod}

\defmethodx{interval}{interval\&}{operator/=}{(const interval\& \var{I})}
\begin{defmethod}{interval}{interval\&}{operator/=}{(double \var{d})}
\begin{operation}
\op{$\code{self}\gets \code{self} / \var{I}$}
\op{$\code{self}\gets \code{self} / \var{d}$}
\end{operation}
Assigns to \code{self} the interval resulting from dividing \code{self} by \var{I}
(resp. \var{d}).
\begin{example}
interval x(-12,12);
x /= interval(-6,23); @rem^// Now, @code^x~ is @itv^-@infty~^+@infty~~
x /= interval::zero;  @rem^^// Now, @code^x~ is @ensuremath^@emptyset~~~
\end{example}
\end{defmethod}

\defmethodx{interval}{interval\&}{operator\%=}{(const interval\& \var{I})}
\begin{defmethod}{interval}{interval\&}{operator\%= }{(double \var{d})}
Assigns to \code{self} the interval resulting from dividing \code{self} by
\var{I} (resp. \var{d}), using a \emph{relational division} (\pxref{sec:relational-arithmetic}).
\begin{example}
interval x(-12,12);
x %= interval(-6,23); @rem^// Now, @code^x~ is @itv^-@infty~^+@infty~~
x %= interval::zero;  @rem^// Now, @code^x~ is @itv^-@infty~^+@infty~~
\end{example}
\end{defmethod}


\chapter{Interval constants}\label{sec:interval-constants}
%%==========================
For convenience, some useful intervals and some canonical intervals enclosing real constants
are defined as static functions of the \code{interval} class:


\begin{center}\renewcommand{\arraystretch}{1.1}
  \begin{tabular}{lc}
    \hline
    \multicolumn{1}{c}{Function} & \multicolumn{1}{c}{Value}\\
    \hline\hline
    \element{interval::emptyset()} & $\emptyset$\\
    \element{interval::half\_pi()} & \itv{\roundDn{\frac{\pi}{2}}}{\roundUp{\frac{\pi}{2}}}\\
    \element{interval::minus\_one\_plus\_one()} & \itv{-1}{1}\\
    \element{interval::negative()} & \itv{-\infty}{0}\\
    \element{interval::one()} & $1$\\
    \element{interval::one\_plus\_infinity()} & \itv{1}{+\infty}\\
    \element{interval::pi()} & \itv{\roundDn{\pi}}{\roundUp{\pi}}\\
    \element{interval::positive()} & \itv{0}{+\infty}\\
    \element{interval::two\_pi()} & \itv{\roundDn{2\pi}}{\roundUp{2\pi}}\\
    \element{interval::universe()} & \itv{-\infty}{+\infty}\\
    \element{interval::zero()} & $0$\\
    \hline
  \end{tabular}
\end{center}

\begin{example}
cout << interval::emptyset;
@outputs^[empty]~
\end{example}


\chapter{Interval relations}\label{sec:interval-relations}
%===========================

Interval relations may be divided into three groups. Given $I$ and
$J$ two intervals, we have:

\begin{enumerate}
\item\dfn{set relations}: intervals $I$ and $J$ are considered as sets of reals. For example:
\begin{equation*}
I=J \Leftrightarrow(\forall x\in I, \exists y\in J\colon x=y)\wedge
                   (\forall y\in J, \exists x\in I\colon x=y)
\end{equation*}
Basically, two intervals are equal in that mode if they have the same bounds;
\item\dfn{certainly relations}: the relations must be true for any tuple of
values in the intervals. For example:
\begin{equation*}
I=J \Leftrightarrow(\forall x\in I, \forall y\in J\colon x=y)
\end{equation*}
Then, two intervals are equal in that mode if they are both reduced to
the same value;
\item\dfn{possibly relations}: the relations are true if it exists at
least one tuple verifying the corresponding real relation. For example:
\begin{equation*}
I=J \Leftrightarrow(\exists x\in I, \exists y\in J\colon x=y)
\end{equation*}
Then, two intervals are equal in that mode whenever their intersection
is not empty.
\end{enumerate}

The kind of relation to associate to relation symbols such as
$==$ and $\leq$ is chosen when configuring the library (\pxref{sec:configuration}).
The other possible definitions are always available through the methods
described hereunder.

\section{Set relations}
%----------------------

\defmethodx{interval}{bool}{set\_contains}{(const interval\& \var{I}) const}
\begin{defmethod}{interval}{bool}{set\_contains}{(double \var{d}) const}
  \begin{operation}
    \op{$I\subseteq\code{self}$}
    \op{$d\in\code{self}$}
  \end{operation}
  Returns \code{true} if \var{I} (resp. \{\var{d}\}) is included in \code{self}.

\begin{example}
interval x(-12,34), y(-12,5);

cout << x.set_contains(y) << endl;
cout << x.set_contains(interval::emptyset) << ' '
     << y.set_contains(x) << ' '
     << interval::emptyset.set_contains(x) << ' '
     << interval::emptyset.set_contains(interval::emptyset)
     << endl;
@outputs^true true false false true~
\end{example}
\end{defmethod}

\defmethodx{interval}{bool}{set\_strictly\_contains}{(const interval\ \var{I}) const}
\begin{defmethod}{interval}{bool}{set\_strictly\_contains}{(double \var{d}) const}
  \begin{operation}
    \op{$I\subset \code{self}$}
    \op{$d\in \code{self}$}
  \end{operation}
Returns \code{true} if \var{I} (resp. \{\var{d}\}) is strictly included in \code{self}

\begin{example}
interval x(-10,12), y(-10, 11), z, t, u(10.5);

cout << boolalpha
    << x.set_strictly_contains(y) << ' '
    << z.strictly_contains(t)
    << ' ' << x.set_strictly_contains(u) << ' '
    << interval::emptyset.set_strictly_contains(
                                   interval::emptyset)
    << ' ' << u.set_strictly_contains(
                          interval::emptyset) << endl;
@outputs^false false true true true~
\end{example}
\end{defmethod}

\begin{defmethod}{interval}{bool}{set\_disjoint}{(const interval\& \var{I}) const}
  \begin{operation}
    \op{$*this\cap I=\emptyset$}
  \end{operation}
Returns \code{true} if the intersection of \code{self} and \var{I} is empty.

\begin{example}
interval a(2,4), b(6,dmax);
cout << a.set_disjoint(b) << " "
     << interval::emptyset.set_disjoint(interval::emptyset);
@outputs^true true~
\end{example}
\end{defmethod}

\begin{defmethod}{interval}{bool}{set\_eq }{(const interval\& \var{I}) const}
  \begin{operation}
\op{$\forall x\in \code{self} \exists y\in I\colon x=y \wedge \forall y\in I, \exists x\in \code{self}\colon y=x$}
  \end{operation}
Returns \code{true} if intervals \code{self} and \var{I} are equal
when considered as sets of reals.
\begin{example}
cout << interval(4,dmax).set_eq(interval(4,dmax)) << " "
     << interval::emptyset.set_eq(interval::empty_set) << endl;
@outputs^true true~
\end{example}
\end{defmethod}

\begin{defmethod}{interval}{bool}{set\_neq}{(const interval\& \var{I}) const}
    \begin{operation}
      \op{$\exists x\in \code{self} \forall y\in I\colon
        x\neq y \vee \exists y\in I \forall x\in \code{self}\colon y\neq x$}
    \end{operation}
    Returns \code{true} if \code{self} and \var{I} are not equal when
    considered as sets of reals.
    \begin{example}
cout << interval::universe.set_neq(interval::emptyset);
@outputs^true~
    \end{example}
\end{defmethod}

\begin{defmethod}{interval}{bool}{set\_le}{(const interval\& \var{I}) const}
  \begin{operation}
    \op{$\forall x \in \code{self}, \exists y\in I\colon x<y \wedge
      \forall y\in I, \exists x \in \code{self}\colon y>x$}
  \end{operation}
Returns \code{true} if the real set defined by \code{self} is strictly included in \code{I}.
\begin{example}
cout << interval(-4.5,3).set_le(interval(-10,10))
     << interval::emptyset.set_le(interval(5,6))
     << interval::emptyset.set_le(interval::emptyset);
@outputs^true false~
\end{example}
\end{defmethod}

\begin{defmethod}{interval}{bool}{set\_leq}{(const interval\& \var{I}) const}
  \begin{operation}
    \op{$\forall x\in \code{self}, \exists y\in I\colon x\leq y \wedge
\forall y\in I, \exists x \in \code{self}\colon y\geq x$}
  \end{operation}
Returns \code{true} if the real set defined by \code{self} is included in \code{I}.
\begin{example}
cout << interval(4.5,6).set_leq(interval(4.5,6))
     << interval(3.5,9).set_leq(interval(2,6))
     << interval::emptyset.set_leq(interval::emptyset);
@outputs^true false true~
\end{example}
\end{defmethod}

\begin{defmethod}{interval}{bool}{set\_ge}{(const interval\& \var{I}) const}
  \begin{operation}
    \op{$\forall x \in \code{self}, \exists y\in I\colon x>y \wedge
      \forall y\in I, \exists x \in \code{self}\colon y<x$}
  \end{operation}
Returns \code{true} if the real set defined by \code{self} strictly contains
\code{I}.
\end{defmethod}

\begin{defmethod}{interval}{bool}{set\_geq}{(const interval\& \var{I}) const}
  \begin{operation}
    \op{$\forall x \in \code{self}, \exists y\in I\colon x\geq y \wedge
      \forall y\in I, \exists x \in \code{self}\colon y\leq x$}
  \end{operation}
Returns \code{true} if the real set defined by \code{self} contains \code{I}.
\end{defmethod}


\section{Certainly relations}
%%---------------------------

\begin{defmethod}{interval}{bool}{certainly\_eq}{(const interval\& \var{I}) const}
  \begin{operation}
    \op{$\forall x\in \code{self}, \forall y \in \var{I}\colon x=y$}
  \end{operation}
  Returns \code{true} if \code{self} is certainly equal to \var{I}, which is true
only when both intervals are degenerate and contain the same floating-point number.
\begin{example}
cout << interval(3,4).certainly_eq(interval(3,4))
     << interval(-6).certainly_eq(interval(-6,-6))
     << interval::universe.certainly_eq(interval::universe)
     << interval::emptyset.certainly_eq(interval::emptyset)
@outputs^false true false true~
\end{example}
\end{defmethod}

\begin{defmethod}{interval}{bool}{certainly\_neq}{(const interval\& \var{I}) const}
  \begin{operation}
    \op{$\forall x\in \code{self}, \forall y \in \var{I}\colon x\neq y$}
  \end{operation}
Returns \code{true} if \code{self} is certainly not equal to \var{I}.
\end{defmethod}

\begin{defmethod}{interval}{bool}{certainly\_le}{(const interval\& \var{I}) const}
  \begin{operation}
    \op{$\forall x\in \code{self}, \forall y \in \var{I}\colon x<y$}
  \end{operation}
  Returns \code{true} if \code{self} is certainly strictly less than \var{I}.
  \begin{example}
cout << interval(4,5).certainly_le(interval(6,9))
     << interval(4,5).certainly_le(interval(5,9))
     << interval::emptyset.certainly_le(interval(4,6));
@outputs^true false true~
  \end{example}
\end{defmethod}

\begin{defmethod}{interval}{bool}{certainly\_leq}{(const interval\& \var{I}) const}
  \begin{operation}
    \op{$\forall x\in \code{self}, \forall y \in \var{I}\colon x\leq y$}
  \end{operation}
  Returns \code{true} if \code{self} is certainly less or equal to \var{I}
  \begin{example}
cout << interval(4,5).certainly_leq(interval(6,9))
     << interval(4,5).certainly_leq(interval(5,9))
     << interval(5,9).certainly_leq(interval(4,5))
     << interval(4,8).certainly_leq(interval(5,9))
     << interval::emptyset.certainly_leq(interval(4,6));
@outputs^true true false false true~
\end{example}
\end{defmethod}

\begin{defmethod}{interval}{bool}{certainly\_ge}{(const interval\& \var{I}) const}
  \begin{operation}
    \op{$\forall x\in \code{self}, \forall y \in \var{I}\colon x>y$}
  \end{operation}
  Returns \code{true} if \code{self} is certainly strictly greater than \var{I}
  \begin{example}
cout << interval(8,10).certainly_ge(interval(4,8))
     << interval::emptyset.certainly_ge(interval::emptyset);
@outputs^false true~
  \end{example}
\end{defmethod}

\begin{defmethod}{interval}{bool}{certainly\_geq}{(const interval\& \var{I}) const}
  \begin{operation}
    \op{$\forall x\in \code{self}, \forall y \in \var{I}\colon x\geq y$}
  \end{operation}
  Returns \code{true} if \code{self} is certainly greater or equal to \var{I}.
  \begin{example}
cout << interval(8,10).certainly_geq(interval(4,8))
     << interval::emptyset.certainly_geq(interval::emptyset);
@outputs^true true~
  \end{example}
\end{defmethod}

\begin{defmethod}{interval}{bool}{certainly\_positive}{(void) const}
  \begin{operation}
    \op{$\code{self}\subseteq\itv{0}{+\infty}$}
  \end{operation}
Returns \code{true} if \code{self} lower bound is greater or equal to zero.
\begin{example}
cout << interval::emptyset.certainly_positive()
     << interval(4,5).certainly_positive()
     << interval(-0.0,6).certainly_positive()
     << interval(-6,0).certainly_positive();
@outputs^true true true false~
\end{example}
\end{defmethod}

\begin{defmethod}{interval}{bool}{certainly\_strictly\_positive}{(void) const}
  \begin{operation}
    \op{$\code{self}\subset\itv{0}{+\infty}$}
  \end{operation}
Returns \code{true} if \code{self} lower bound is strictly greater than zero.
\begin{example}
cout << interval::emptyset.certainly_strictly_positive()
     << interval(4,5).certainly_strictly_positive()
     << interval(-0.0,6).certainly_strictly_positive()
     << interval(-6,0).certainly_strictly_positive();
@outputs^true true false false~
\end{example}
\end{defmethod}

\begin{defmethod}{interval}{bool}{certainly\_negative}{(void) const}
  \begin{operation}
    \op{$\code{self}\subseteq\itv{-\infty}{0}$}
  \end{operation}
Returns \code{true} if \code{self} lower bound is lower or equal to zero.
\begin{example}
cout << interval::emptyset.certainly_negative()
     << interval(4,5).certainly_negative()
     << interval(-6,0).certainly_negative()
     << interval(-6,-5).certainly_negative();
@outputs^true false true true~
\end{example}
\end{defmethod}

\begin{defmethod}{interval}{bool}{certainly\_strictly\_negative}{(void) const}
  \begin{operation}
    \op{$\code{self}\subset\itv{-\infty}{0}$}
  \end{operation}
Returns \code{true} if \code{self} lower bound is strictly lower than zero.
\begin{example}
cout << interval::emptyset.certainly_strictly_negative()
     << interval(4,5).certainly_strictly_negative()
     << interval(-6,0).certainly_strictly_negative()
     << interval(-6,-5).certainly_strictly_negative();
@outputs^true false false true~
\end{example}
\end{defmethod}

\section{Possibly relations}
%%--------------------------

\begin{defmethod}{interval}{bool}{possibly\_eq}{(const interval\& \var{I}) const}
  \begin{operation}
    \op{$\exists x\in \code{self}, \exists y \in \var{I}\colon x=y$}
  \end{operation}
Returns \code{true} if \code{self} is possibly equal to \var{I}.
\begin{example}
cout << interval(5,10).possibly_eq(interval(6,100))
     << interval::emptyset.possibly_eq(interval::emptyset);
@outputs^true false~
\end{example}
\end{defmethod}

\begin{defmethod}{interval}{bool}{possibly\_neq}{(const interval\& \var{I}) const}
  \begin{operation}
    \op{$\exists x\in \code{self}, \exists y \in \var{I}\colon x\neq y$}
  \end{operation}
Returns \code{true} if \code{self} is possibly not equal to \var{I}
\begin{example}
cout << interval(4,5).possibly_neq(interval(4,5))
     << interval(4,4).possibly_eq(interval(4,4))
     << interval::emptyset.possibly_eq(interval::emptyset);
@outputs^true false false~
\end{example}
\end{defmethod}

\begin{defmethod}{interval}{bool}{possibly\_le}{(const interval\& \var{I}) const}
  \begin{operation}
    \op{$\exists x\in \code{self}, \exists y \in \var{I}\colon x<y$}
  \end{operation}
Returns \code{true} if \code{self} is possibly strictly less than \var{I}.
\begin{example}
cout << interval(4,5).possibly_le(interval(3,7))
     << interval(4,5).possibly_le(interval(2,4))
     << ;interval(4,5).possibly_le(interval::emptyset)
@outputs^true false false~
\end{example}
\end{defmethod}

\begin{defmethod}{interval}{bool}{possibly\_leq}{(const interval\& \var{I}) const}
  \begin{operation}
    \op{$\exists x\in \code{self}, \exists y \in \var{I}\colon x\leq y$}
  \end{operation}
  Returns \code{true} if \code{self} is possibly less or equal to \var{I}.
\begin{example}
cout << interval(4,5).possibly_leq(interval(3,7))
     << interval(4,5).possibly_leq(interval(2,4))
     << ;interval(4,5).possibly_leq(interval::emptyset)
@outputs^true true false~
\end{example}
\end{defmethod}

\begin{defmethod}{interval}{bool}{possibly\_ge}{(const interval\& \var{I}) const}
  \begin{operation}
    \op{$\exists x\in \code{self}, \exists y \in \var{I}\colon x>y$}
  \end{operation}
  Returns \code{true} if \code{self} is possibly strictly greater than \var{I}
  \begin{example}
cout << interval(4,5).possibly_ge(interval(3,6))
     << interval(4,5).possibly_ge(interval(5,6))
     << interval(4,5).possibly_ge(interval(6,7))
     << interval(4,5).possibly_ge(interval::emptyset);
@outputs^true false false false false~
  \end{example}
\end{defmethod}

\begin{defmethod}{interval}{bool}{possibly\_geq}{(const interval\& \var{I}) const}
  \begin{operation}
    \op{$\exists x\in \code{self}, \exists y \in \var{I}\colon x\geq y$}
  \end{operation}
  Returns \code{true} if \code{self} is possibly greater or equal to \var{I}.
  \begin{example}
cout << interval(4,5).possibly_geq(interval(3,6))
     << interval(4,5).possibly_geq(interval(5,6))
     << interval(4,5).possibly_geq(interval(6,7))
     << interval(4,5).possibly_geq(interval::emptyset);
@outputs^true true false false false~
  \end{example}
\end{defmethod}

\section{Relational Symbols}
%%--------------------------

\begin{deffun}{bool}{operator==}{(const interval\& \var{I1}, const interval\& \var{I2})}
    Returns \code{\var{I1}.set\_eq(\var{I2})},
    \code{\var{I1}.certainly\_eq(\var{I2})}, or
    \code{\var{I1}.possibly\_eq(\var{I2})} depending on the default kind of
    relation chosen when configuring gaol (\pxref{sec:configuration}).
\end{deffun}

\begin{deffun}{bool}{operator!=}{(const interval\& \var{I1}, const interval\& \var{I2})}
  Returns \code{\var{I1}.set\_neq(\var{I2})},
  \code{\var{I1}.certainly\_neq(\var{I2})}, or
  \code{\var{I1}.possibly\_neq(\var{I2})} depending on the default kind of
  relation chosen when configuring gaol (\pxref{sec:configuration}).
\end{deffun}

\begin{deffun}{bool}{operator<}{(const interval\& \var{I1}, const interval\& \var{I2})}
  Returns \code{\var{I1}.set\_le(\var{I2})},
  \code{\var{I1}.certainly\_le(\var{I2})}, or
  \code{\var{I1}.possibly\_le(\var{I2})} depending on the default kind of
  relation chosen when configuring gaol (\pxref{sec:configuration}).
\end{deffun}

\begin{deffun}{bool}{operator<=}{(const interval\& \var{I1}, const interval\& \var{I2})}
  Returns \code{\var{I1}.set\_leq(\var{I2})},
  \code{\var{I1}.certainly\_leq(\var{I2})}, or
  \code{\var{I1}.possibly\_leq(\var{I2})} depending on the default kind of
  relation chosen when configuring gaol (\pxref{sec:configuration}).
\end{deffun}

\begin{deffun}{bool}{operator>}{(const interval\& \var{I1}, const interval\& \var{I2})}
  Returns \code{\var{I1}.set\_ge(\var{I2})},
  \code{\var{I1}.certainly\_ge(\var{I2})}, or
  \code{\var{I1}.possibly\_ge(\var{I2})} depending on the default kind of
  relation chosen when configuring gaol (\pxref{sec:configuration}).
\end{deffun}

\begin{deffun}{bool}{operator>=}{(const interval\& \var{I1}, const interval\& \var{I2})}
  Returns \code{\var{I1}.set\_geq(\var{I2})},
  \code{\var{I1}.certainly\_geq(\var{I2})}, or
  \code{\var{I1}.possibly\_geq(\var{I2})} depending on the default kind of
  relation chosen when configuring gaol (\pxref{sec:configuration}).
\end{deffun}


\section{Interval-specific relations}
%%-----------------------------------

\begin{defmethod}{interval}{bool}{straddles\_zero}{(void) const}
  \begin{operation}
    \op{$0 \in \code{self}$}
  \end{operation}
Returns \code{true} if \code{self} contains zero.
\begin{example}
interval x(0,4), y, z(-12,-5);

cout << boolalpha << x.straddles_zero() << ' '
     << y.straddles_zero() << ' '
     << z.straddles_zero() << endl;
@outputs^true true false~
\end{example}

\paragraph*{Note.} \code{I.straddles\_zero()} $\equiv$ \code{I.set\_contains(interval::Real(0))}
\end{defmethod}

\begin{defmethod}{interval}{bool}{strictly\_straddles\_zero}{(void) const}
  \begin{operation}
  \op{$\{0\}\subset\code{self}$}
  \end{operation}
  Returns \code{true} if zero is included in the interior of \code{self}.
\begin{example}
interval x(0,4), y, z(-12,-5);

cout << boolalpha << x.strictly_straddles_zero() << ' '
     << y.strictly_straddles_zero() << ' '
     << z.strictly_straddles_zero() << endl;
@outputs^false true false~
\end{example}
\paragraph*{Note.} The method call to \code{I.strictly\_straddles\_zero()} is equivalent to
                   \code{I.set\_strictly\_contains(interval::Real(0))}
\end{defmethod}


\begin{defmethod}{interval}{bool}{is\_a\_double}{(void) const}
Returns \code{true} whenever the left and right bounds of the interval
are equal.
\begin{example}
interval x(-12.5), y;

cout << boolalpha << x.is_a_double() << ' '
     << y.is_a_double()
     << interval::emptyset.is_a_double() << endl;
@outputs^true false false~
\end{example}
\end{defmethod}

\begin{defmethod}{interval}{bool}{is\_an\_int}{(void) const}
  Returns \code{true} whenever the left and right bounds of the interval
  are equal and castable into an integer (type \code{int}).
\begin{example}
interval x(-12.0), y;

cout << boolalpha << x.is_an_int() << ' ' << y.is_an_int()
     << interval::emptyset.is_an_int() << endl;
@outputs^true false false~
\end{example}
\end{defmethod}


\begin{defmethod}{interval}{bool}{is\_canonical}{(void) const}
  Returns \code{true} if \code{self} contains at most two floating-point
  numbers.
\begin{example}
interval x(0), y(-14,9);

cout << boolalpha << x.is_canonical()      << ' ' <<
                     y.is_canonical()      << ' ' <<
                     interval::emptyset << ' ' <<
                     interval::pi;
@outputs^true false false true~
\end{example}
\end{defmethod}

\begin{defmethod}{interval}{bool}{is\_empty}{(void) const}
  \begin{operation}
    \op{$\code{self} = \emptyset$}
  \end{operation}
  Returns \code{true} if \code{self} is an empty interval.
  \begin{example}
cout << interval(4,5).is_empty()
     << interval(5,4).is_empty()
     << interval::emptyset.is_empty();
@outputs^false true true~
  \end{example}
\end{defmethod}

\begin{defmethod}{interval}{bool}{is\_zero}{(void) const}
  \begin{operation}
    \op{$\code{self} = \itv{0}{0}$}
  \end{operation}
  Returns \code{true} if \code{self} is equal to the interval containing only $0$.
  \begin{example}
cout << interval::zero.is_zero()
     << interval(0.0,0.0).is_zero()
     << interval(-0.0,+0.0).is_zero()
     << interval(0,5).is_zero()
     << interval::emptyset.is_zero();
@outputs^true true true false false~
  \end{example}
\end{defmethod}

\begin{defmethod}{interval}{bool}{is\_symmetric}{(void) const}
  Returns \code{true} if the left bound of \code{self} is the opposite of the
  right bound. An empty interval is not symmetric.

\begin{example}
cout << interval(-5,5).is_symmetric()
     << interval::emptyset.is_symmetric() << endl;
@outputs^true false~
\end{example}
\end{defmethod}

\begin{defmethod}{interval}{bool}{is\_finite}{(void) const}
  Returns \code{true} if both bounds are finite.

\begin{example}
cout << interval("[4,inf]").is_finite()
     << interval::emptyset.is_finite()
     << interval(5,80).is_finite()
@outputs^false true true~
\end{example}
\end{defmethod}

\chapter{Interval Arithmetic}
%%===========================

The \emph{containment principle} of (floating-point) interval arithmetic
imposes that for any operation ``$\circ$'', and any intervals \var{I} and \var{J},
the following does hold:
\begin{equation*}
I\circ J= \hull{\{i\circ j\mid i\in I, j\in J\}}
\end{equation*}
where \hull{} is a function mapping any real set to the
smallest floating-point interval containing it.

For example, if we consider the interval square root, we have:
\begin{equation*}
\sqrt{\var{I}} = \hull{\{\sqrt{i}\mid x\in \var{I}\}}
\end{equation*}

From monotonicity considerations, the square root of \itv{1}{2} is then
$\sqrt{\itv{1}{2}}=\itv{1}{\sqrt{2}}$. Now, another
interpretation of the square root function is as follows:

\begin{equation*}
\sqrt{_r\var{I}} = \hull{\{j\in\RSet\mid\exists i\in \var{I}\colon j^2=i\}}
\end{equation*}

This last definition stands for the \dfn{relational square root} and
permits obtaining both negative and positive values. Hence, we have:
\begin{equation*}
\sqrt{_r\itv{1}{2}} = \itv{-\sqrt{2}}{\sqrt{2}}
\end{equation*}

This operator arises when we consider the relation
\begin{equation*}
x^2= y
\end{equation*}
which can alternatively be written
\begin{equation*}
x=\sqrt{_r y}
\end{equation*}
Here, the \dfn{functional square root} is not suitable since it would
induce the intersection of the domain of $y$ with \itv{0}{+\infty}.

Some applications (mainly in the area of \emph{constraint programming})
require the availability of such operators. As a consequence, gaol offers
both functional and relational versions of the main arithmetic operators.

\section{Functional Arithmetic}
%%-----------------------------

\defmethodx{interval}{interval}{operator+}{(void) const}
\deffunx{interval}{operator+}{(const interval\& \var{I}, double \var{d})}
\deffunx{interval}{operator+}{(double \var{d}, const interval\& \var{I})}
\begin{deffun}{interval}{operator+}{(const interval\& \var{I1}, const interval\& \var{I2})}
  Addition of two intervals, or of one interval and a double.
\end{deffun}

\defmethodx{interval}{interval}{operator-}{(void) const}
\deffunx{interval}{operator-}{(const interval\& \var{I}, double \var{d})}
\deffunx{interval}{operator-}{(double \var{d}, const interval\& \var{I})}
\begin{deffun}{interval}{operator-}{(const interval\& \var{I1}, const interval\& \var{I2})}
  Negation, or subtraction of two intervals, or of one interval and a double.
\end{deffun}


\deffunx{interval}{operator*}{(const interval\& \var{I}, double \var{d})}
\deffunx{interval}{operator*}{(double \var{d}, const interval\& \var{I})}
\begin{deffun}{interval}{operator*}{(const interval\& \var{I1}, const interval\& \var{I2})}
  Multiplication of two intervals, or of one interval and a double.
\end{deffun}

\deffunx{interval}{operator/}{(const interval\& \var{I}, double \var{d})}
\deffunx{interval}{operator/}{(double \var{d}, const interval\& \var{I})}
\begin{deffun}{interval}{operator/}{(const interval\& \var{I1}, const interval\& \var{I2})}
  Functional division of two intervals, or of one interval and a double.
\end{deffun}

\deffunx{interval}{operator\%}{(const interval\& \var{I}, double \var{d})}
\deffunx{interval}{operator\%}{(double \var{d}, const interval\& \var{I})}
\begin{deffun}{interval}{operator\%}{(const interval\& \var{I1}, const interval\& \var{I2})}
  Relational division of two intervals, or of one interval and a double.
\end{deffun}

\begin{deffun}{interval}{sqrt}{(const interval\& \var{I})}
  \begin{operation}
    \op{$\sqrt{I}$}
  \end{operation}
  Functional square root of \var{I}.
\end{deffun}

\begin{deffun}{interval}{sqr}{(const interval\& \var{I})}
  \begin{operation}
    \op{$\var{I}^2$}
  \end{operation}
  Square of \var{I}
\end{deffun}

\deffunx{interval}{pow}{(const interval\& \var{I}, int \var{b})}
\begin{deffun}{interval}{pow}{(const interval\& \var{I1}, const interval\& \var{I2})}
  \begin{operation}
    \op{$I^b$}
    \op{$I_1^{I_2}$}
  \end{operation}
  Power function. The former computes \var{I} to the integral power \var{b},
  while the latter raises \var{I1} to the interval power \var{I2}. If
  \var{I2} is an \code{int} in disguise, the first function is used to
  improve accuracy.

  The power function for interval exponents is not yet fully tested and should be used with care.
\end{deffun}

\begin{deffun}{interval}{nth\_root}{(const interval\& \var{I}, int \var{b})}
  \begin{operation}
    \op{$^{b}\sqrt{I}$}
  \end{operation}
  Computes the \var{b}th functional root of \var{I}.
\end{deffun}


\begin{deffun}{interval}{exp}{(const interval\& \var{I})}
  Exponential of \var{I}.
\end{deffun}

\begin{deffun}{interval}{log}{(const interval\& \var{I})}
  Natural logarithm of \var{I}.
\end{deffun}

\subsection{Trigonometric functions}
%%..................................

\begin{deffun}{interval}{cos}{(const interval\& \var{I})}
  Returns the cosine of \var{I}.
\end{deffun}

\begin{deffun}{interval}{acos}{(const interval\& \var{I})}
  Returns the arccosine of \var{I}.
\end{deffun}

\begin{deffun}{interval}{sin}{(const interval\& \var{I})}
  Returns the sine of \var{I}.
\end{deffun}

\begin{deffun}{interval}{asin}{(const interval\& \var{I})}
  Returns the arcsine of \var{I}.
\end{deffun}

\begin{deffun}{interval}{tan}{(const interval\& \var{I})}
  Returns the tangent of \var{I}.
\end{deffun}

\begin{deffun}{interval}{atan}{(const interval\& \var{I})}
  Returns the arctangent of \var{I}.
\end{deffun}

\subsection{Hyperbolic functions}
%%...............................

\begin{deffun}{interval}{cosh}{(const interval\& \var{I})}
  Returns the hyperbolic cosine of \var{I}.
\end{deffun}

\begin{deffun}{interval}{acosh}{(const interval\& \var{I})}
  Returns the hyperbolic arccosine of \var{I}.
\end{deffun}

\begin{deffun}{interval}{sinh}{(const interval\& \var{I})}
  Returns the hyperbolic sine of \var{I}.
\end{deffun}

\begin{deffun}{interval}{asinh}{(const interval\& \var{I})}
  Returns the hyperbolic arcsine of \var{I}.
\end{deffun}

\begin{deffun}{interval}{tanh}{(const interval\& \var{I})}
  Returns the hyperbolic tangent of \var{I}.
\end{deffun}

\begin{deffun}{interval}{atanh}{(const interval\& \var{I})}
  Returns the hyperbolic arctangent of \var{I}.
\end{deffun}

\section{Relational Arithmetic}\label{sec:relational-arithmetic}
%%-----------------------------

\defmethodx{interval}{interval}{operator\% }{(double \var{d}) const}
\defmethodx{interval}{interval}{operator\% }{(const interval\& \var{I}) const}
\begin{deffun}{interval}{operator\%}{(double \var{d}, const interval\& \var{I})}
  \begin{operation}
    \op{$*this/\var{I}=\{z\in \RSet\mid\exists x\in *this, \exists y\in I\colon x=yz\}$}
  \end{operation}
Relational division.
\end{deffun}

\subsection{$(n+1)$-ary relational functions}
%%...........................................

Consider the relation $y = \cos x$ where $x$ and $y$ are
interval variables. One would like to be able to express this relation
in the equivalent way: $x = \mathop{acos} y$. However, one cannot use the
$\mathop{acos}$ function because its result is always included into the
interval \itv{0}{\pi}. What we need here is a relational
version of the $\mathop{acos}$ function. But, since for any value $x$
there are infinitely many values $y$ verifying $x=\mathop{acos} y$, we
have to take into account the domain of $y$. As a consequence, we
define a new binary operator \code{acos\_rel} whose definition is as
follows:

\begin{equation*}
{\tt\mathop{acos\_rel}}(Y,X)=\hull{\{x\in X\mid\exists y\in Y\colon y=\cos x\}}
\end{equation*}

\noindent
This is to be contrasted with the previous definition of the
$\mathop{acos}$ function:

\begin{equation*}
\code{acos}(Y)=\hull{\{x\in\RSet\mid\exists y\in Y\colon x=\mathop{acos} y\}}
\end{equation*}

\noindent
Figure~\ref{fig:relational-cos} presents the different results obtained when computing
either $\code{acos}(J)$ or $\code{acos\_rel}(J,I)$.

\begin{wrapfigure}{O}{\marginparwidth}
 \includegraphics[width=.9\marginparwidth,bb=11 262 553 470]{relation-cos}
  \caption{Relational cosine}
  \label{fig:relational-cos}
\end{wrapfigure}

\begin{deffun}{interval}{acos\_rel}{(const interval\& J, const interval\& I)}
  \begin{operation}
    \op{$\code{acos\_rel}(J,I)=\hull{\{x\in I\mid\exists y\in J\colon y=\cos x\}}$}
  \end{operation}
  Returns the \emph{relational arccosine} of \var{J} w.r.t. \var{I}.
\end{deffun}

\begin{deffun}{interval}{asin\_rel}{(const interval\& J, const interval\& I)}
  \begin{operation}
    \op{$\code{asin\_rel}(J,I)=\hull{\{x\in I\mid\exists y\in J\colon y=\sin x\}}$}
  \end{operation}
  Returns the \emph{relational arcsine} of \var{J} w.r.t. \var{I}.
\end{deffun}

\begin{deffun}{interval}{atan\_rel}{(const interval\& J, const interval\& I)}
  \begin{operation}
    \op{$\code{atan\_rel}(J,I)=\hull{\{x\in I\mid\exists y\in J\colon y=\tan x\}}$}
  \end{operation}
  Returns the \emph{relational arctangent} of \var{J} w.r.t. \var{I}.
\end{deffun}

\begin{deffun}{interval}{acosh\_rel}{(const interval\& J, const interval\& I)}
  \begin{operation}
    \op{$\code{acosh\_rel}(J,I)=\hull{\{x\in I\mid\exists y\in J\colon y=\cosh x\}}$}
  \end{operation}
  Returns the \emph{relational hyperbolic arccosine} of \var{J} w.r.t. \var{I}.
\end{deffun}

\begin{deffun}{interval}{sqrt\_rel}{(const interval\& J, const interval\& I)}
  \begin{operation}
    \op{$\code{sqrt\_rel}(J,I)=\hull{\{x\in I\mid\exists y\in J\colon y=x^2\}}$}
  \end{operation}
  Returns the \emph{relational square root} of \var{J} w.r.t. \var{I}.
\end{deffun}

\begin{deffun}{interval}{nth\_root\_rel}{(const interval\& \var{J}, unsigned int \var{n}, \\\hbox{}\hfill const interval\& \var{I})}
  \begin{operation}
    \op{$\code{nth\_root\_rel}(J,n,I)=\hull{\{x\in I\mid\exists y\in J\colon y=x^n\}}$}
  \end{operation}
  Returns the \emph{relational inverse \var{b}-th root} of
  \var{J} w.r.t. \var{I}.
\end{deffun}

\begin{deffun}{interval}{invabs\_rel}{(const interval\& J, const interval\& I)}
  \begin{operation}
    \op{$\code{invabs\_rel}(J,I)=\hull{\{x\in I\mid |x|\in J\}}$}
  \end{operation}
  Returns the \emph{relational inverse absolute value} of \var{J} w.r.t. \var{I}.
\end{deffun}

\begin{deffun}{interval}{div\_rel}{(const interval\& K, const interval\& J, const interval\& I)}
  \begin{operation}
    \op{$\code{div\_rel}(K,J,I)=\hull{\{x\in I\mid\exists z\in K,\exists y\in J\colon z=xy\}}$}
  \end{operation}
  Returns the \emph{ternary relational division} of \var{K} by \var{J}
  w.r.t. \var{I}.
\end{deffun}

Relational function that do not appear here are identical to their inverse function (e.g., the
relational hyperbolic arcsine is identical to the hyperbolic arcsine, which should be used instead).


\chapter{Interval functions}
%%==========================

\begin{defmethod}{interval}{double}{width}{(void) const}
  \begin{operation}
    \op{$\code{width}(\itv{a}{b})=\roundUp{b-a}$}
  \end{operation}
Returns the width of \code{self}. Returns \code{-1.0} whenever the interval is empty.
\begin{example}
cout << interval(4,6).width()
     << (interval(1,next_float(1)).width()
         == std::numeric_limits<double>::epsilon())
     << interval::emptyset.width();
@outputs^2 true -1~
\end{example}
\end{defmethod}

\marginnocite{Hansen:book92} % To handle the minipage in environment 'defmethod'
\begin{defmethod}{interval}{double}{mig}{(void) const}
  Returns the \dfn{mignitude} of \code{self}. See the book by
  Hansen~\margincite*[chap.~3]{Hansen:book92}. The mignitude of an
  interval \itv{a}{b} is the smallest absolute value of the numbers in
  the interval, that is: $0$ if the interval straddles \code{0}, $a$
  if the interval is strictly positive, and $-b$ otherwise.

\paragraph{Note.} The mignitude of the empty interval is a NaN.
\begin{example}
cout << interval(4,5).mig()
     << interval(-6,-3).mig()
     << interval(-3,8).mig();
@outputs^4 3 0~
\end{example}
\end{defmethod}

\marginnocite{Stahl:PhD}
\begin{defmethod}{interval}{double}{smig}{(void) const}
  Returns the \dfn{signed mignitude} of \code{self}.  See Stahl's
  thesis~\margincite*[def.~1.3.28]{Stahl:PhD}. The signed mignitude of
  an interval \itv{a}{b} is $0$ if the interval straddles \code{0},
  $a$ if the interval is strictly positive, and $b$ otherwise.

\paragraph{Note.} The signed mignitude of the empty interval is an NaN.
\begin{example}
cout << interval(4,5).mig()
     << interval(-6,-3).mig()
     << interval(-3,8).mig();
@outputs^4 -3 0~
\end{example}
\end{defmethod}

\begin{defmethod}{interval}{double}{mag}{(void) const}
  Returns the \dfn{magnitude} of \code{self}. the magnitude of an
  interval \itv{a}{b} is the greatest absolute value of the numbers in
  the interval.

\paragraph{Note.} The magnitude of the empty interval is a NaN.
\begin{example}
cout << interval(4,5).mag()
     << interval(-6,-3).mag()
     << interval(-10,5).mag();
@outputs^5 6 10~
\end{example}
\end{defmethod}

\begin{deffun}{double}{hausdorff}{(const interval\& \var{I1}, const interval\& \var{I2})}
  Returns the Hausdorff distance between the two sets defined by intervals \var{I1} and
  \var{I2}, that is:
\begin{equation*}
\code{hausdorff}(I_1,I_2)=\max(|\leftBound{I_1}-\leftBound{I_2}|,
                          |\rightBound{I_1}-\rightBound{I_2}|)
\end{equation*}
\begin{example}
cout << hausdorff(interval(4,8),interval(5,10))
@outputs^2~
\end{example}
\end{deffun}

\begin{defmethod}{interval}{double}{midpoint}{(void) const}
  Returns the midpoint of \code{self}. Given $a$ and $b$ two finite
  floating-point numbers and \code{std::numeric\_limits<double>::max()} the largest positive
  floating-point number of type double, we have the following cases:
\begin{equation*}
\left\{\begin{array}{ll}
\code{midpoint}(\emptyset{})&=\text{NaN}\\
\code{midpoint}(\itv{-\infty}{+\infty})&=0\\
\code{midpoint}(\itv{-\infty}{b})&=-\code{std::numeric\_limits<double>::max()}\\
\code{midpoint}(\itv{a}{+\infty})&=\code{std::numeric\_limits<double>::max()}\\
\code{midpoint}(\itv{a}{b})&=\roundNearest{(a+b)/2} \\
\end{array}\right.
\end{equation*}
\end{defmethod}

\begin{defmethod}{interval}{interval}{mid}{(void) const}
  Returns an interval enclosing the midpoint of \code{self}. The result is not
  guaranteed to be canonical though it is always included in \code{self}.
  With the same notations as for \code{midpoint()}, we have the cases:
\begin{equation*}
\left\{\begin{array}{ll}
\code{mid}(\emptyset{})&=\emptyset{}\\
\code{mid}(\itv{-\infty}{+\infty})&=\itv{0}{0}\\
\code{mid}(\itv{-\infty}{b})&=[-\code{std::numeric\_limits<double>::max()}]\\
\code{mid}(\itv{a}{+\infty})&=[\code{std::numeric\_limits<double>::max()}]\\
\code{mid}(\itv{a}{b})&=\itv{\roundDn{(a+b)/2}}{\roundUp{(a+b)/2}}
\end{array}\right.
\end{equation*}
\end{defmethod}

\begin{defmethod}{interval}{double}{left}{(void) const}
  \begin{operation}
    \op{$\code{left}(\itv{x}{y})=x$}
  \end{operation}
  Returns the left bound of \code{self}. Note that this method may return a finite
  floating-point number (i.e. neither a NaN, nor an infinity) even when the
  interval itself is empty.
\end{defmethod}


\begin{defmethod}{interval}{double}{right}{(void) const}
  \begin{operation}
    \op{$\code{right}(\itv{x}{y})=y$}
  \end{operation}
  Returns the right bound of \code{self}. Note that this method may return a finite
  floating-point number (i.e. neither a NaN, nor an infinity) even when the
  interval itself is empty.
\end{defmethod}

\begin{deffun}{interval}{abs}{(const interval\& \var{I})}
  \begin{operation}
    \op{$\text{abs}(I)=\{|x|\in\RSet^+\mid x\in I\}$}
  \end{operation}
  Returns the absolute value of \var{I}.
  \begin{example}
cout << abs(interval(-5,6))
     << abs(interval(-4,-2));
@outputs^@itv^0~^6~ @itv^2~^4~~
  \end{example}
\end{deffun}

\marginnocite{Ratschek-Rokne:1995}
\begin{deffun}{double}{chi}{(const interval \&\var{I})}
This function, introduced by Ratscheck and Rokne~\margincite*{Ratschek-Rokne:1995},
characterizes the degree of symmetry of intervals. Its definition is as follows:

\begin{equation*}
\text{For} I=\itv{a}{b}, \quad\text{chi}(I)=\left\{\begin{array}{ll}
    -1  & \text{if } I = 0\\
    a/b & \text{if } |a|\leq |b|\\
    b/a & \text{otherwise}
  \end{array}\right.
\end{equation*}
\begin{example}
cout << chi(interval(3,6))
     << chi(interval(-6,3))
     << chi(interval::emptyset)
     << chi(interval::universe)
     << chi(interval("[-5,inf]"));
@outputs^0.5 -0.5 NaN 1 0~
\end{example}
\end{deffun}

\begin{deffun}{interval}{min}{(const interval \&I, const interval \&J)}
  \begin{operation}
    \op{$\min(\itv{a}{b}, \itv{c}{d}) = \itv{\min(a,c)}{\min(b,d)}$}
  \end{operation}
  Returns the minimum of two intervals.
  \begin{example}
cout << min(interval(5,6),interval(3,9))
     << min(interval::emptyset,interval(3,8));
@outputs^@itv^3~^6~ @text^[empty]~~
  \end{example}
\end{deffun}

\begin{deffun}{interval}{max}{(const interval \&\var{I}, const interval \&\var{J})}
  \begin{operation}
    \op{$\max(\itv{a}{b}, \itv{c}{d}) = \itv{\max(a,c)}{\max(b,d)}$}
  \end{operation}
  Returns the maximum of two intervals.
\begin{example}
cout << max(interval(5,6),interval(3,9))
     << max(interval::emptyset,interval(3,8));
@outputs^@itv^5~^9~ @text^[empty]~~
\end{example}
\end{deffun}

\begin{deffun}{interval}{floor}{(const interval \&\var{I})}
  \begin{operation}
    \op{$\itv{\code{floor}(I.\code{left}())}{\code{floor}(I.\code{right}())}$}
  \end{operation}

  \begin{example}
cout << floor(interval(4.5,6.5))
     << floor(interval("[-10.4,3.5]"));
@outputs^@itv^4~^6~ @itv^-11~^3~~
  \end{example}
\end{deffun}

\begin{deffun}{interval}{ceil}{(const interval \&\var{I})}
  \begin{operation}
    \op{\itv{\code{ceil}(I.\code{left}())}{\code{ceil}(I.\code{right}())}}
  \end{operation}
  \begin{example}
cout << ceil(interval(4.5,6.5))
     << ceil(interval("[-10.4,3.5]"));
@outputs^@itv^5~^7~ @itv^-10~^4~~
  \end{example}
\end{deffun}

\begin{deffun}{interval}{integer}{(const interval \&\var{I})}
  \begin{operation}
    \op{\itv{\code{ceil}(I.\code{left}())}{\code{floor}(I.\code{right}())}}
  \end{operation}
  Narrows down the bounds to the closest integers. Note that the
  resulting bounds are still \code{double} numbers, and may therefore
  not be representable with integral types.
  \begin{example}
cout << integer(interval(4.5,6.5));
@outputs^@itv^5~^6~~
  \end{example}
\end{deffun}


\section{Splitting methods}
%%-------------------------

\begin{defmethod}{interval}{void}{split}{(interval\& \var{I1}, interval\& \var{I2}) const}
  Splits \code{self} into two parts using \code{midpoint()}; returns the left part
  in \var{I1} and the right part in \var{I2}.

  \var{I1} or \var{I2} may be equal to \code{self}.
  \begin{example}
interval I1a, I2a,
         I3(1.0,next_float(1.0)),
         I1b, I2b;
interval(4,5).split(I1a,I2a);
I3.split(I1b,I2b);
cout << I1a << " " << I2a << " "
     << (I1b==1.0) << " " << (I2b==I3) << endl;
@outputs^@itv^4~^4.5~ @itv^4.5~^5~ true true~
  \end{example}
\end{defmethod}

\begin{defmethod}{interval}{interval}{split\_left}{(void) const}
  \begin{operation}
    \op{$\code{split\_left}(\itv{a}{b})=\itv{a}{\roundUp{(a+b)/2}}$}
  \end{operation}
  Splits \code{self} into two parts using \code{midpoint()} and returns the left
  part.
  \begin{example}
cout << interval(4,5).split_left();
@outputs^@itv^4~^4.5~~
  \end{example}
\end{defmethod}

\begin{defmethod}{interval}{interval}{split\_right}{(void) const}
  \begin{operation}
    \op{$\code{split\_right}(\itv{a}{b})=\itv{\roundUp{(a+b)/2}}{b}$}
  \end{operation}
  Splits \code{self} into two parts using \code{midpoint()} and
  returns the right part.

\paragraph{Note.} The left bound of the result is rounded up such that there is
the least overlap possible with the interval returned by \code{split\_left()}.
  \begin{example}
cout << interval(4,5).split_left();
@outputs^@itv^4.5~^5~~
  \end{example}
\end{defmethod}

\section{Union and intersection}
%%------------------------------

\begin{deffun}{interval}{operator\&}{(const interval\& I1, const interval\& I2)}
  \begin{operation}
    \op{$\var{I1}\cap\var{I2}$}
  \end{operation}
  Returns the interval resulting from the intersection of \var{I1} and
  \var{I2}.
  \begin{example}
cout << interval(4,6) & interval(5,9);
@outputs^@itv^5~^6~~
  \end{example}
\end{deffun}

\begin{deffun}{interval}{operator|}{(const interval\& I1, const interval\& I2)}
  \begin{operation}
    \op{$\var{I1}\cup\var{I2}$}
  \end{operation}
  Returns the interval resulting from the union of \var{I1} and \var{I2}.
  \begin{example}
cout << interval(3,6) | interval(9,12);
@outputs^@itv^3~^12~~
  \end{example}
\end{deffun}

\chapter{Input/output}
%%====================

\section{Reading intervals}\label{sec:reading-intervals}
%%-------------------------

\begin{deffun}{istream\&}{operator>{}>}{(ostream\& \var{in}, interval\& I)}
  Reads an interval from the input stream \var{in} and assigns it to
  \var{I}.  If the string read is syntactically ill-formed, an
  \code{input\_format\_error} exception is thrown
  (\pxref{sec:exceptions}) if the library was compiled with exceptions
  enabled (\pxref{sec:configuration}); alternatively, it prints an
  error message to \code{cerr} and aborts if exceptions were disabled.
\end{deffun}

\subsection{Input format}\label{sec:input-format}
%%.......................

A string to be translated into an interval must have the following
syntax (with terminals in lower case and non-terminals in slanted upper case):

\begin{Verbatim}[commandchars=\@\^\~]
@NT^itv_expr~
        : @NT^parsed_interval~
	| @NT^itv_expr~ @TM^+~ @NT^itv_expr~
	| @NT^itv_expr~ @TM^-~ @NT^itv_expr~
	| @NT^itv_expr~ @TM^*~ @NT^itv_expr~
	| @NT^itv_expr~ @TM^/~ @NT^itv_expr~
	| @TM^-~ @NT^itv_expr~
	| @TM^+~ @NT^itv_expr~
	| @NT^itv_function_call~
	| @TM^(~ @NT^itv_expr~ @TM^)~
	;

@NT^itv_function_call~
	: @TM^cos~ @TM^(~ @NT^itv_expr~ @TM^)~
	| @TM^sin~ @TM^(~ @NT^itv_exprt~ @TM^)~
	| @TM^tan~ @TM^(~ @NT^itv_expr~ @TM^)~
	| @TM^atan2~ @TM^(~ @NT^itv_expr~ @TM^,~ @NT^itv_expr~ @TM^)~
	| @TM^acos~ @TM^(~ @NT^itv_expr~ @TM^)~
	| @TM^asin~ @TM^(~ @NT^itv_expr~ @TM^)~
	| @TM^atan~ @TM^(~ @NT^itv_expr~ @TM^)~
	| @TM^cosh~ @TM^(~ @NT^itv_expr~ @TM^)~
	| @TM^sinh~ @TM^(~ @NT^itv_expr~ @TM^)~
	| @TM^tanh~ @TM^(~ @NT^itv_expr~ @TM^)~
	| @TM^acosh~ @TM^(~ @NT^itv_expr~ @TM^)~
	| @TM^asinh~ @TM^(~ @NT^itv_expr~ @TM^)~
	| @TM^atanh~ @TM^(~ @NT^itv_expr~ @TM^)~
	| @TM^exp~ @TM^(~ @NT^itv_expr~ @TM^)~
	| @TM^log~ @TM^(~ @NT^itv_expr~ @TM^)~
	| @TM^pow~ @TM^(~ @NT^itv_expr~ @TM^,~ @NT^itv_expr~ @TM^)~
	| @TM^sqrt~ @TM^(~ @NT^itv_expr~ @TM^)~
	| @TM^nth_root~ @TM^(~ @NT^itv_expr~ @TM^,~ @NT^itv_expr~ @TM^)~
	;

@NT^parsed_interval~
	: @NT^expression~
	| @TM^empty~ @rem^@hfill// Empty interval~
	| [ @NT^expression~ ]
	| [ @NT^expression~ @TM^,~ @NT^expression~ ]
	| < @NT^expression~ @TM^,~ @NT^expression~ >
	| [ @TM^empty~ ] @rem^@hfill// Empty interval~
	;

@NT^expression~
	: @NT^number~
	| @TM^dmin~  @rem^@hfill// Smallest positive floating-point number~
	| @TM^dmax~ @rem^@hfill// Largest positive floating-point number~
	| @TM^pi~
	| @TM^inf~ @rem^@hfill// Floating-point positive ``infinity''~
	| @NT^expression~ + @NT^expression~
	| @NT^expression~ - @NT^expression~
	| @NT^expression~ * @NT^expression~
	| @NT^expression~ / @NT^expression~
	| - @NT^expression~
	| + @NT^expression~
	| @NT^function_call~
	| @TM^(~ @NT^expression~ @TM^)~
	;

@NT^function_call~
	: @TM^cos~ @TM^(~ @NT^expression~ @TM^)~
	| @TM^sin~ @TM^(~ @NT^expression~ @TM^)~
	| @TM^tan~ @TM^(~ @NT^expression~ @TM^)~
	| @TM^atan2~ @TM^(~ @NT^expression~ @TM^,~ @NT^expression~ @TM^)~
	| @TM^acos~ @TM^(~ @NT^expression~ @TM^)~
	| @TM^asin~ @TM^(~ @NT^expression~ @TM^)~
	| @TM^atan~ @TM^(~ @NT^expression~ @TM^)~
	| @TM^cosh~ @TM^(~ @NT^expression~ @TM^)~
	| @TM^sinh~ @TM^(~ @NT^expression~ @TM^)~
	| @TM^tanh~ @TM^(~ @NT^expression~ @TM^)~
	| @TM^acosh~ @TM^(~ @NT^expression~ @TM^)~
	| @TM^asinh~ @TM^(~ @NT^expression~ @TM^)~
	| @TM^atanh~ @TM^(~ @NT^expression~ @TM^)~
	| @TM^exp~ @TM^(~ @NT^expression~ @TM^)~
	| @TM^log~ @TM^(~ @NT^expression~ @TM^)~
	| @TM^pow~ @TM^(~ @NT^expression~ @TM^,~ @NT^expression~ @TM^)~
	| @TM^sqrt~ @TM^(~ @NT^expression~ @TM^)~
	| @TM^nth_root~ @TM^(~ @NT^expression~ @TM^,~ @NT^expression~ @TM^)~
	;
\end{Verbatim}

Spaces are not significant except in numbers. The ``$+$'' sign before
numbers and \code{inf} is optional.
Note that the second argument of \code{nth\_root} shall be a point interval that can be evaluated
as an integer. Otherwise, an \element{invalid\_action\_error} exception is thrown (or an error is 
reported, depending on whether exceptions were enabled or not at configuration time---see Section~\ref{sec:configuration}, page~\pageref{sec:configuration}).

If a rational number is not representable in the floating-point format, it is
replaced by the smallest floating-point interval containing it. The notations
"$n$" and "[$n$]" are equivalent.

The two bounds in the string ``\element{<a, b>}'' must be expressions that evaluate to the same value even for different rounding directions. An
\element{input\_format\_error} exception is raised otherwise.

\begin{example}
interval x("[4, 6*7]");
interval y("[-inf, dmax]");
interval z("[3.14,3.15]/8", "[3.14,3.15]/7");
interval t("[3.14,3.15]/[7,8]");
\end{example}

\paragraph*{Caution 1:} Case is significant for all the operators.

\paragraph*{Caution 2:} Numbers appearing in the string shall not have more than 
15 digits, otherwise their translation to floating-point numbers is not 
guaranteed to be correct.

Expressions in bounds are evaluated using interval arithmetic; the left (resp. right)
bound is then used, depending on the side it appeared in.

Note that, as of version~\version, the \code{atan2} operator is not yet implemented for interval
expressions.

Contrary to constructors that take numeric constants as parameters, expressions such as \element{"inf"} or \element{"<-inf, -inf>"} lead to legitimate intervals (see the note Page~\pageref{par:inf_inf_note}).

\section{Writing intervals}
%%-------------------------

Intervals may be printed into a stream like any other C++ primitive type
by using the ``\code{<{}<}'' operator.

\begin{deffun}{ostream\&}{operator<{}<}{(ostream\& \var{out}, const interval\& I)}
Prints the interval \var{I} to the output stream \var{out}. The way the intervals
are actually displayed depends on the active format (see next section).
However, whatever the format, an empty interval is always displayed as \code{[empty]}
\cindex{empty!interval: printing}%
\index[idx]{empty@\code{[empty]}}%
\end{deffun}

\subsection{Converting intervals to strings}
%%..........................................

For convenience, the \code{interval} class provides a conversion operator
into the standard C++ type \code{string}.
\cindex{interval!conversion to \code{string}}%
\index[idx]{string@\code{string}!converting interval to|see{interval}}

\begin{example}
  interval I(3,4);
  string s = "test line embedding " + string(I) + " as a string";
  @rem^// Now, s is "test line embedding [3, 4] as a string"~
\end{example}

\subsection{Output format}\label{sec:output-format}
%%........................

Intervals may be displayed following four different formats:
\begin{description}
\item[\element{agreeing}.] By printing all the digits that are the same in the left and right bounds
followed by an interval containing the remaining digits:
\begin{center}
``\element{3.141\textasciitilde[5926, 6001]}'' stands for
``\element{\itv{3.1415926}{3.1416001}}''
\end{center}
Note that if the bounds do not have any agreeing digit, there will still be a tilde before the bracketed part:
\begin{example}
 interval::format(interval_format::agreeing);
 std::cout << interval(4,6) << std::endl;
 @outputs^@textasciitilde[4., 6.]~
\end{example}

\item[\element{bounds}.] By printing the bounds between square brackets.
Degenerate intervals whose left and right bound are equal are printed with angles:
\begin{example}
 interval::format(interval_format::bounds);
 std::cout << interval(3,5) << std::endl;
 @outputs^[3, 5]~
 std::cout << interval(4) << std::endl;
 @outputs^<4, 4>~
\end{example}

\item[\element{width}.] by printing their midpoint and their width:
\begin{example}
 interval::format(interval_format::width);
 std::cout << interval(4,5) << std::endl;
 @outputs^4.5 (+/- 0.5)~
\end{example}

\item[\element{hexa}.] by printing the hexadecimal representation of their left and right
bounds (useful when one wants to know the precise value of the bound without being
affected by the round-off error due to binary-to-decimal conversion):
\begin{example}
 interval::format(interval_format::hexa);
 std::cout << interval("0.1") << std::endl;
 @outputs^[3fb9999999999999, 3fb999999999999a]~
\end{example}

\end{description}

\medskip
Note that \emph{the \element{bounds} format is the only one recognized at present as an input} (see previous section).
\medskip

The choice of the format to use is made through the following static methods:

\defstaticmethodx{interval}{void}{format}{(interval\_format::format\_t \var{f})}
\begin{defstaticmethod}{interval}{interval\_format::format\_t}{format}{(void)}
  The first form of the method allows modifying the format to use in subsequent
  printing of intervals. The second form reports what is the current form in use.
  It returns a value of  type \code{interval\_format::format\_t}
  (see below and \ref{sec:output-example} for an example of use).
\end{defstaticmethod}

\begin{deftype}{struct}{interval\_format}
  Structure type used to choose the ouput format for intervals. It has
  four possible values of type \element{interval\_format::format\_t}:
\begin{itemize}
\item \element{interval\_format::agreeing}.
\item \element{interval\_format::bounds}.
\item \element{interval\_format::width}.
\item \element{interval\_format::center}.
\item \element{interval\_format::hexa}.
\end{itemize}
\end{deftype}

\begin{example}
  interval I(interval::pi);

  interval::format(interval_format::agreeing);
  cout << I << "\n";
  @rem^// Prints 3.14159265358979@textasciitilde[3, 4]~
  @rem^// The @textasciitilde[] part is dropped if the bounds agree on all digits~
  @smallskip
  interval::format(interval_format::bounds);
  cout << I << "\n";
  @rem^// Prints [3.141592653589793, 3.141592653589794]~
  @smallskip
  interval::format(interval_format::width);
  cout << I << "\n";
  @rem^// Prints 3.141592653589793 (+/- 2.220446049250313e-16)~
  @smallskip
  interval::format(interval_format::center);
  cout << I << "\n";
  @rem^// Prints 3.141592653589793~
  @smallskip
  interval::format(interval_format::hexa);
  cout << I << "\n";
  @rem^// Prints [400921fb54442d18, 400921fb54442d19]~
\end{example}

\subsection{Choosing the number of digits to display}
%%...................................................
You can manipulate the number of digits to print by using the \code{precision()} static
methods of the interval class:

\begin{defmethod}{interval}{std::streamsize}{precision}{(void)}
Returns the current number of digits used for printing bounds
of intervals.

See example below.
\end{defmethod}

\begin{defmethod}{interval}{std::streamsize}{precision}{(std::streamsize \var{n})}
Set the number of digits to use for printing bounds to \var{n}. In addition,
returns the number of digits previously used.

See example below.
\end{defmethod}


\subsection{Example}\label{sec:output-example}
%%..................

\begin{example}
#include <iostream>
#include <gaol/gaol.h>

using std::cout;
using std::endl;

int main(void)
{
  gaol::init();
  interval::precision(4);
  interval::format(interval_format::bounds);
  cout << interval::pi << endl;

  if (interval::format() != interval_format::bounds) {
    cout << interval::pi << endl;
  } else {
    int old_prec = interval::precision(16);
    interval::format(interval_format::width);
    cout << interval::pi << endl;
  }
  gaol::cleanup();
}
\end{example}

\noindent
On a Pentium-based PC, the previous program has the following output:

\begin{onscreen}
[ 3.142, 3.142 ]
3.141592653589793 (+/- 2.220446049250313e-16)
\end{onscreen}

The first call to \code{interval::format()} is unnecessary
since the default format is \code{interval\_format::bounds}.

\paragraph*{Note.} Translating an interval into a string and then reading it back
as an interval is likely to produce an inaccurate or plain wrong result if you
choose a precision different from \code{17}. It is however useless to specify a
precision greater than \code{17} for the \code{double} format since the extra
digits would be garbage.

The \code{interval::pi} constant is a predefined \dfn{canonical}
interval containing $\pi$ (\pxref{sec:interval-constants}). Here,
the width of the interval is equal to the $\epsilon$ of
the format.

The \code{interval\_format::width} format may be useful whenever the number of
digits displayed is insufficient to know whether the result is a single
floating-point number or an interval whose size is very small (consider
for example the first result above), because \emph{we have the guarantee that
if the actual width of an interval is greater than zero, the width
displayed will also be different from zero}. Another indication is that a degenerate
interval is displayed as a floating-point number.

\chapter{Floating-point numbers}
%%==============================

\section{Floating-point constants}\label{sec:floating-point-constants}
%%--------------------------------

In addition to the constants available through \code{numeric\_limits<double>},
gaol defines the following \code{double} constants:

\begin{center}
  \renewcommand{\arraystretch}{1.1}
  \begin{tabular}{lc}
    \hline
    \multicolumn{1}{c}{\bfseries Constant (\texttt{double})} & \multicolumn{1}{c}{\bfseries Value}\\
    \hline\hline
    \element{two\_pi} & \roundNearest{2\pi}\\
    \element{pi} & \roundNearest{\pi}\\
    \element{half\_pi} & \roundNearest{\frac{\pi}{2}}\\
    \element{pi\_dn} & \roundDn{\pi}\\
    \element{pi\_up} & \roundUp{\pi}\\
    \element{half\_pi\_dn} & \roundDn{\frac{\pi}{2}}\\
    \element{half\_pi\_up} & \roundUp{\frac{\pi}{2}}\\
    \element{ln2\_dn} & \roundDn{\ln 2}\\
    \element{ln2\_up} & \roundUp{\ln 2}\\
    \element{two\_power\_53} & $2^{53}$\\
    \element{GAOL\_NAN} & NaN (quiet)\\
    \element{GAOL\_INFINITY} & $+\infty$ \\
    \hline
  \end{tabular}
\end{center}

\section{Floating-point functions}
%%--------------------------------

\begin{deffun}{bool}{feven}{(const double\& \var{x})}
  Returns true whenever \var{x} is even.

  This function should not be used with infinity and NaN arguments.

  \begin{example}
    assert( feven(3.0) ); @rem^@hfill// false~
    assert( feven(3.5) ); @rem^@hfill// false~
    assert( feven(4.0) ); @rem^@hfill// true~
    assert( feven(4.5) ); @rem^@hfill// false~
    assert( feven(GAOL_INFINITY) ); @rem^@hfill// always true~
    assert( feven(GAOL_NAN) ); @rem^@hfill// always false~
  \end{example}
\end{deffun}

\begin{deffun}{double}{next\_float}{(double \var{x})}
  Returns the smallest \code{double} greater than \var{x}.
\end{deffun}

\begin{deffun}{double}{previous\_float}{(double \var{x})}
  Returns the greatest \code{double} smaller than \var{x}.
\end{deffun}

\begin{deffun}{bool}{is\_signed}{(double \var{x})}
Returns true whenever \var{x} is signed. No provision is made concerning the
fact that \var{x} is a NaN. If you only want to test for negative
numbers (and $-0$), you will have to test also whether \var{x} is a NaN
by using the \code{isnan()} predicate in \code{math.h}.
\end{deffun}

\begin{deffun}{double}{minimum}{(double \var{x}, double \var{y})}
  Returns the minimum \code{double} value of \var{x}
  and \var{y}. This function is commutative and returns $-0$ when comparing
  $-0$ and $+0$, i.e.:
\begin{equation*}
\left\{\begin{array}{lll}
    \min(x,y) &=\min(y,x), &\forall x\neq\text{NaN}, \forall y\neq\text{NaN}\\
    \min(x,\text{NaN}) &=\min(\text{NaN},x)=\text{NaN}, &\forall x\\
    \min(-0,0) &=\min(0,-0)=-0
  \end{array}\right.
\end{equation*}
\end{deffun}

\begin{deffun}{double}{maximum}{(double \var{x},double \var{y})}
  Returns the maximum \code{double} value of \var{x}
  and \var{y}. This function is commutative and returns $+0$ when comparing
  $-0$ and $+0$, i.e.:
\begin{equation*}
  \left\{\begin{array}{lll}
      \max(x,y) &= \max(y,x), &\forall x\neq\text{NaN}, \forall y\neq\text{NaN} \\
      \max(x,\text{NaN})&=\max(\text{NaN},x)=\text{NaN}, &\forall x\\
      \max(-0,0)&=\max(0,-0)=0
    \end{array}\right.
\end{equation*}
\end{deffun}

\begin{defmacro}{ULONGLONGINT}
  Macro standing for an unsigned integral data type with a size equal to 8 bytes
  (usually \code{unsigned long long int}).
\end{defmacro}

\begin{deffun}{ULONGLONGINT}{nb\_fp\_numbers}{(double \var{a}, double \var{b})}
Returns the number of floating-point numbers in the interval
\itv{\var{a}}{\var{b}}. In particular, we have:
\begin{itemize}
\item \code{nb\_fp\_numbers(a,next\_float(a)) == 2}
\item \code{nb\_fp\_numbers(a,a) == 1}
\end{itemize}

\paragraph{Note.} As a precondition, \var{a} shall be lower or equal to \var{b}.

Returns \code{numeric\_limits<ULONGLONGINT>::max()} if either
\var{a} or \var{b} is a NaN or an infinity. In addition, raises
an \code{invalid\_action\_error} exception (\pxref{sec:exceptions}) or calls
\code{gaol\_error} depending on the way the library was configured.
\end{deffun}

\chapter{Manipulating the FPU}
%%============================

The gaol library provides functions to manipulate the FPU and its
flags. The main functions are the one described in the next section
for modifying the rounding direction.  As for now, gaol provides these
facilities for the following platforms:

\begin{itemize}
\item ix86 and compatibles under Linux
\item SPARC under Solaris
\item ISO C99-compliant platforms
\end{itemize}

Whenever possible, inline assembler versions are used.

\section{Rounding functions}
%%--------------------------

\begin{deffun}{void}{round\_downward}{(void)}
Sets the rounding direction mode towards $-\infty$.
\end{deffun}

\begin{deffun}{void}{round\_nearest}{(void)}
  Sets the rounding direction mode to the nearest/even.
\end{deffun}

\begin{deffun}{void}{round\_zero}{(void)}
  Sets the rounding direction mode to zero.
\end{deffun}

\begin{deffun}{void}{round\_upward}{(void)}
  Sets the rounding direction mode to $+\infty$.
\end{deffun}

\section{Manipulating the FPU flags}
%%----------------------------------

The following functions allow to manipulate the FPU flags. See the documentation
of the FPU for your machine for a description of these flags.

\begin{deffun}{void}{clear\_inexact}{(void)}
  Clears the \code{inexact} flag of the FPU.

\paragraph{Warning:} This function is currently unavailable on some platforms.
For these platforms, a warning is issued when the function is called.
\end{deffun}


\begin{deffun}{int}{get\_inexact}{(void)}
Returns a non-zero value whenever the last floating-point operation was performed
with rounding. The associated FPU flag is a persistent one. As a consequence,
you should always clear it by calling \code{clear\_inexact()}
\emph{before} performing the operation you want to test.
\end{deffun}

\begin{deffun}{unsigned short}{get\_fpu\_cw}{(void)}
  Returns the value of the FPU control word.

\paragraph{Warning:} This function is currently unavailable on some platforms.
For these platforms, a warning is issued when the function is called.
\end{deffun}

\begin{deffun}{unsigned short}{get\_fpu\_sw}{(void)}
Returns the value of the FPU status word.

\paragraph{Warning:} This function is currently unavailable on some platforms.
For these platforms, a warning is issued when the function is called.
\end{deffun}

\chapter{Version information}
%%===========================

The library provides four constants to allow programs to determine at runtime with which version
they are dynamically linked with. The versioning scheme adopted is the one used by the Apache
Software Foundation described at \url{http://apr.apache.org/versioning.html}.

\begin{defconst}{unsigned int}{version\_major}
  Major version of the library.
\end{defconst}

\begin{defconst}{unsigned int}{version\_minor}
  Minor version of the library.
\end{defconst}

\begin{defconst}{unsigned int}{version\_patch}
  Patch version of the library.
\end{defconst}

\begin{defconst}{const char *const}{version}
  Version of the library as a string.
\begin{example}
 const char *const version = "1.0.3";
\end{example}
\end{defconst}

\chapter{Additional functions}
%%============================

The following functions are utility functions not necessarily related to
intervals or floating-point numbers.

\begin{defTfun}{template <typename T>}{bool}{odd}{(const \var{T}\& \var{x})}
  Returns \code{true} if \var{x} is odd and false otherwise. The \var{T}
  type may be any type providing the \code{\&} (``bitwise \code{and}'') operator
  with the same semantics as the one for \code{int}s.
\end{defTfun}

\begin{defTfun}{template <typename T>}{bool}{even}{(const \var{T}\& \var{x})}
  Returns \code{true} if \var{x} is even and false otherwise. The \var{T}
  type may be any type providing the \code{\&} (``bitwise \code{and}'') operator
  with the same semantics as the one for \code{int}s.
\end{defTfun}

\chapter{Error handling}
%%======================

A program that uses gaol may report errors in two different ways:
\begin{itemize}
\item by throwing an exception;
\item or by setting the \code{errno} variable.
\end{itemize}

The mecanism in use depends on the way the library is configured. If
you use the option \option{-{}-enable-exceptions=yes}, all errors are
reported through exception throwing; otherwise, the \code{errno}
variable is used. Relying on exceptions is more in the C++ spirit,
though it may incur some overhead.

It is up to the user to comply with this mecanism when adding error reporting
code to ones program. Gaol defines the following macro to be used
whenever one wants to report an error.

\begin{defmacroA}{gaol\_ERROR}{(\var{excep},\var{msg})}
  The behavior of the macro depends on the value chosen for the option
  \option{-{}-enable-exceptions}: if exceptions are enabled, exception
  \var{excep} is raised with the message \var{msg}; otherwise, the
  program aborts with message \var{msg}.

\begin{example}
interval x;

[Code manipulating x]

if (x.is_empty()) {
   gaol_ERROR(failure_error,"Emptyness of one interval");
}
\end{example}
\end{defmacroA}

The \code{gaol\_error()} function is defined as follows:

\deffunx{void}{gaol\_error}{(const char *const \var{err})}
\begin{deffun}{void}{gaol\_error}{(const char *\var{file}, int \var{line},
const char *\var{err})}
Displays a message on the standard error output. The ternary version should be
called with the \code{GAOL\_FILE\_POS} macro for the first two parameters.
\end{deffun}

The \code{GAOL\_FILE\_POS} macro is described in the next section.

\section{Exceptions}\label{sec:exceptions}
%%------------------

The library defines \code{gaol\_exception} as a class to be used as a base
class for all gaol exceptions.
All of them provide at least the name of the file and the line number from
where the exception has been thrown. As a facility, gaol defines the following
macro:

\begin{defmacro}{GAOL\_FILE\_POS}
Expands itself into the first two arguments of any constructor for
\code{gaol\_exception} or one of its derived classes:

\begin{example}
if (@emph^[some condition]~) {
  throw gaol_exception(GAOL_FILE_POS,
                       "No additional information");
}
\end{example}
\end{defmacro}

All gaol exceptions can be sent to an output stream through the ``\code{<{}<}''
operator.

\subsection{The \code{gaol\_exception} exception}\label{sec:gaol_exception}
%%...............................................
The \code{gaol\_exception} class is the base class from which derive all
gaol exceptions. It inherits from the C++ standard class \code{exception}.

Every exception class deriving from it must at least provide the name of
the file and the line where the corresponding exception was thrown. As a
consequence, the constructors for \code{gaol\_exception} are as follows:

\defmethodx{gaol\_exception}{}{gaol\_exception}{(const char* \var{f}, unsigned \var{l})}
\begin{defmethod}{gaol\_exception}{}{gaol\_exception}{(const char* \var{f}, unsigned \var{l}, \\\hbox{}\hfill const char* \var{e})}
Constructs a \code{gaol\_exception} being thrown from file \var{f} at
line \var{l}. The second form permits adding some explanatory string \var{e}.
\end{defmethod}

\noindent
The class offers the following accessors:

\begin{defmethod}{gaol\_exception}{const char*}{file}{(void) const}
  Return the name of the file from where the exception was thrown.
\end{defmethod}

\begin{defmethod}{gaol\_exception}{unsigned int}{line}{(void) const}
  Returns the line number in the file from where the exception was thrown.
\end{defmethod}

\begin{defmethod}{gaol\_exception}{const char* const}{explanation}{(void) const}
  Returns a string explaining why the exception was thrown. Returns an empty
  string if no additional information was provided.
\end{defmethod}

\subsection{The \code{input\_format\_error} exception}
%%....................................................
The \code{input\_format\_error} exception is thrown whenever one attempts to
create an interval from an invalid string. This situation may occur when
reading an interval from a stream with the \code{>{}>} operator, or when
creating an interval from a string.

This class, as all gaol exceptions, derives from \code{gaol\_exception}
(\pxref{sec:gaol_exception}). Its constructors have the same format
than the ones for \code{gaol\_exception}, namely:

\defmethodx{input\_format\_error}{}{input\_format\_error}{(const char* \var{f}, unsigned \var{l})}
\begin{defmethod}{input\_format\_error}{}{input\_format\_error}{(const char* \var{f}, unsigned \var{l}, const char* \var{e})}
Constructs an \code{input\_format\_error} being thrown from file \var{f} at
line \var{l}. The second form permits adding some explanatory string \var{e}.
\end{defmethod}
The methods of the class are inherited from \code{gaol\_exception}
(\pxref{sec:gaol_exception}).

\subsection{The \code{unavailable\_feature\_error} exception}
%%---------------------------------------------------------

This exception is thrown whenever an unavailable feature is requested.

This class, as all gaol exceptions, derives from \code{gaol\_exception}.
(\pxref{sec:gaol_exception}). Its constructors have the same format
than the ones for \code{gaol\_exception}, namely:

\defmethodx{unavailable\_feature\_error}{}{unavailable\_feature\_error}{\\\hbox{}\hfill (const char* \var{f}, unsigned \var{l})}
\begin{defmethod}{unavailable\_feature\_error}{}{unavailable\_feature\_error}{\\\hbox{}\hfill (const char* \var{f}, unsigned \var{l}, const char* \var{e})}
Constructs an \code{unavailable\_feature\_error} being thrown from file \var{f} at
line \var{l}. The second form permits adding some explanatory string \var{e}.
\end{defmethod}

The methods of the class are inherited from \code{gaol\_exception} (\pxref{sec:gaol_exception}).

\subsection{The \code{invalid\_action\_error} exception}
%%------------------------------------------------------

This exception is thrown whenever a function is called with invalid arguments
(e.g. calling \code{nb\_fp\_numbers()} with NaNs as parameters).

This class, as all gaol exceptions, derives from \code{gaol\_exception}
(\pxref{sec:gaol_exception}). Its constructors have the same format
than the ones for \code{gaol\_exception}, namely:

\defmethodx{invalid\_action\_error}{}{invalid\_action\_error}{(const char* \var{f},\\\hbox{}\hfill  unsigned \var{l})}
\begin{defmethod}{invalid\_action\_error}{}{invalid\_action\_error}{(const char* \var{f},\\\hbox{}\hfill  unsigned \var{l}, const char* \var{e})}
Constructs an \code{invalid\_action\_error} being thrown from file \var{f} at
line \var{l}. The second form permits adding some explanatory string \var{e}.
\end{defmethod}

The methods of the class are inherited from \code{gaol\_exception} (\pxref{sec:gaol_exception}).


\section{Warnings}
%%----------------

\deffunx{void}{gaol\_warning}{(const char *\var{warn})}
\begin{deffun}{void}{gaol\_warning}{(const char *\var{file}, int \var{line}, const char *\var{warn})}
Prints the message \var{warn} on the standard error output. The second form
should be called with the \code{GAOL\_FILE\_POS} macro for the first two parameters.
\end{deffun}

\chapter{Debugging facilities}\label{sec:debugging-facilities}
%%============================

The debugging facilities described hereunder are available only if
gaol has been configured with the debugging facilities enabled (see
the \option{-{}-enable-debug} option, Section~\ref{sec:configuration},
p.~\pageref{sec:configuration}).

\begin{deffun}{int}{debug\_level}{}
  Global variable used to remember the current value of the debugging
  level.  This variable is set when initializing the library. The
  variable is declared in the \code{gaol} namespace.
\end{deffun}

\begin{defmacroA}{GAOL\_DEBUG}{(\var{lvl},\var{cmd})}
  Executes \var{cmd} if \var{lvl} is lower or equal to the current debugging
  level (see the variable \code{debug\_level} above).

  This macro defaults to nothing if the library was not configured with the
\option{-{}-enable-debug} option.
\end{defmacroA}

\noindent
A possible use for this macro is as follows:

\begin{example}
interval x(-10,10);

@emph^[Some code]~

GAOL_DEBUG(1,cout << "The value of x is " << x);
x += double_interval(3.5,4.5);
GAOL_DEBUG(2,cout << "Now the value of x is " << x);
\end{example}

The first message will be displayed whenever gaol has been configured with
debugging facilities enabled (see the \option{-{}-enable-debug} option). The
second message will be displayed only if the debugging level
is greater or equal to 2.

\begin{defmacro}{GAOL\_ASSERT}{(\var{cond})}
Tests whether \var{cond} holds. Aborts with an error message if it is not the case.

This macro defaults to nothing if the library was not configured with the
\option{-{}-enable-debug} option.
\end{defmacro}

\noindent
A possible use for this macro is as follows:

\begin{example}
int x;
cout << "Give an integer no greater than 5: ";
cin >> x;
GAOL_ASSERT(x <= 5);
\end{example}

\chapter{Profiling}
%%=================

The following functions permit computing the time used for a computation. The
returned times are \emph{user} times, meaning that delays induced by
input/output operations and freezing during CPU switches in multi-programming
environments are not taken into account.

If you need to keep track of several events, consider using an object of the
\code{timepiece} class (\pxref{sec:timepiece}) instead of calling
directly the functions below.

\paragraph*{Warning.} The precision of the timing functions depends on the platform
used. For example, the precision on ix86-based machines is usually no better
than 10 ms. What is more, despite the fact that the reported times are user
times, they may vary from an execution to another, and can get larger on
heavily loaded machines.


\begin{deffun}{long}{get\_time}{(void)}
Returns the time in milliseconds since a certain unspecified moment.

This function should only be used to compute differences between two
calls since the starting point may vary depending on the availability of
\code{clock()} or \code{getrusage()} on the system.


\paragraph*{Warning.} if the function in use is the standard \code{clock()},
the time returned will wrap approximately every 72 minutes. Consequently, it is
not safe to use \code{get\_time()} in that case for processes requiring more than
72 minutes to execute.
\end{deffun}


\begin{deffun}{void}{reset\_time}{(void)}
  Resets the time counter. To be called just before executing some code to be
  profiled.
\end{deffun}

\begin{deffun}{long}{elapsed\_time}{(void)}
  Returns the time in milliseconds elapsed between now and the last call to
  \code{reset\_time()}.
\end{deffun}

\begin{deffun}{long}{intermediate\_elapsed\_time}{(void)}
  Returns the time in milliseconds elapsed between now and the last call to
  \code{reset\_time()} or to \code{intermediate\_elapsed\_time()}.
\end{deffun}

Here is a typical example of use of the timing functions:

\begin{example}
int main(void)
{
  reset_time();
  for (unsigned int i=0;i<1000;++i) {
     @emph^[Some time consuming operations]~
  }
  cout << "Elapsed time: " << elapsed_time() << " ms." << endl;
  return 0;
}
\end{example}

\section{The \code{timepiece} class}\label{sec:timepiece}
%%----------------------------------

A \code{timepiece} object allows to keep track of the time spent to perform
a particular task. Since the counter used is local to the object, it is possible
to monitor more than one such process.

\subsection{Methods of the \code{timepiece} class}
%%................................................

\begin{defmethod}{timepiece}{void}{start}{(void)}
  Starts the timepiece.
\end{defmethod}

\begin{defmethod}{timepiece}{void}{stop}{(void)}
  Stops the timepiece and accumulates the time spent since the last call to \code{start()}.
\end{defmethod}

\begin{defmethod}{timepiece}{void}{reset}{(void)}
  Resets to zero the counter keeping track of the total time the timepiece was running.
\end{defmethod}

\begin{defmethod}{timepiece}{long}{get\_total\_time}{(void) const}
  Returns the total amount of time the timepiece was running (time between calls to
  \code{start()} and \code{stops}. The timepiece shall have been stopped by calling the
  \code{stop()} method before calling this one.
\end{defmethod}

\begin{defmethod}{timepiece}{long}{get\_intermediate\_time}{(void) const}
  Returns the amount of time spent since the last call to \code{start()}.
\end{defmethod}


\begin{example}
int main(void)
{
  timepiece t;
  t.start();
  for (unsigned int i=0;i<1000;++i) {
     @emph^[Some time consuming operations]~
     cout << "Intermediate time: "
          << t.get_intermediate_time() << " ms." << endl;
  }
  t.stop();
  cout << "Elapsed time: " << t.get_total_time() << " ms." << endl;
  return 0;
}
\end{example}


\chapter{Additional Documentation}
%%================================

\section{Documentation on gaol}
%%-----------------------------

The primary reference is this manual. There is also an html reference
for the code itself, which might be of interest only to developers
seeking to understand/modify gaol.

\section{References}
%%------------------
The following articles and books have inspired in some way or another the
devising of the gaol library and/or the writing of this manual.

\begin{itemize}
\item \emph{Interval Arithmetic Specification}. Dmitri Chiriaev and G.\:
William Walster. Draft revised May 1998.
\item \emph{The Extended Real Interval System}. G.\: William
Walster. April 1998.
\item \emph{C++ Interval Arithmetic Programming Reference}. Sun Microsystems,
Inc. October 2000, revision A.
\item \emph{Interval Arithmetic: From Principles to Implementation}. T.\:
Hickey, Q.\: Ju, and M.\: H.\: van Emden. Tech. Rep. CS-99-202,
CS Dept. Brandeis U, July 1999.
\end{itemize}

\chapter{Reporting bugs}
%%======================

All bugs and suggestions for improvement shall be submitted through the appropriate form
available on the web site:

\begin{center}
  \url{http://sourceforge.net/projects/gaol/}
\end{center}

\chapter{Contributors}
%%====================

The main implementor and lead designer for the gaol library is Fr\'ed\'eric Goualard
(\href{mailto:goualard@users.sourceforge.net}{goualard@users.sourceforge.net}).

The \code{interval pow(const interval\&, const interval\&)} function
was designed by Marc Christie
(\href{mailto:christie@users.sourceforge.net}{christie@users.sourceforge.net}).

The code for the multiplication and the division is largely inspired
from the one presented by Tim Hickey, Qun Ju and Maarten Van Emden in
\emph{Interval Arithmetic: from Principles to Implementation} Journal
of the ACM 48(5):1038--1068, september 2001.


\schapter{Library Copying}\label{chap:copying}
%%========================
\centerline{\textbf{GNU LIBRARY GENERAL PUBLIC LICENSE}}
\centerline{Version 2, June 1991}

Copyright \copyright\ 1991 Free Software Foundation, Inc.
675 Mass Ave, Cambridge, MA 02139, USA

Everyone is permitted to copy and distribute verbatim copies
of this license document, but changing it is not allowed.

[This is the first released version of the library GPL\@.  It is
 numbered 2 because it goes with version 2 of the ordinary GPL.]

\section*{Preamble}

  The licenses for most software are designed to take away your
freedom to share and change it.  By contrast, the GNU General Public
Licenses are intended to guarantee your freedom to share and change
free software--to make sure the software is free for all its users.

  This license, the Library General Public License, applies to some
specially designated Free Software Foundation software, and to any
other libraries whose authors decide to use it.  You can use it for
your libraries, too.

  When we speak of free software, we are referring to freedom, not
price.  Our General Public Licenses are designed to make sure that you
have the freedom to distribute copies of free software (and charge for
this service if you wish), that you receive source code or can get it
if you want it, that you can change the software or use pieces of it
in new free programs; and that you know you can do these things.

  To protect your rights, we need to make restrictions that forbid
anyone to deny you these rights or to ask you to surrender the rights.
These restrictions translate to certain responsibilities for you if
you distribute copies of the library, or if you modify it.

  For example, if you distribute copies of the library, whether gratis
or for a fee, you must give the recipients all the rights that we gave
you.  You must make sure that they, too, receive or can get the source
code.  If you link a program with the library, you must provide
complete object files to the recipients so that they can relink them
with the library, after making changes to the library and recompiling
it.  And you must show them these terms so they know their rights.

  Our method of protecting your rights has two steps: (1) copyright
the library, and (2) offer you this license which gives you legal
permission to copy, distribute and/or modify the library.

  Also, for each distributor's protection, we want to make certain
that everyone understands that there is no warranty for this free
library.  If the library is modified by someone else and passed on, we
want its recipients to know that what they have is not the original
version, so that any problems introduced by others will not reflect on
the original authors' reputations.

  Finally, any free program is threatened constantly by software
patents.  We wish to avoid the danger that companies distributing free
software will individually obtain patent licenses, thus in effect
transforming the program into proprietary software.  To prevent this,
we have made it clear that any patent must be licensed for everyone's
free use or not licensed at all.

  Most GNU software, including some libraries, is covered by the ordinary
GNU General Public License, which was designed for utility programs.  This
license, the GNU Library General Public License, applies to certain
designated libraries.  This license is quite different from the ordinary
one; be sure to read it in full, and don't assume that anything in it is
the same as in the ordinary license.

  The reason we have a separate public license for some libraries is that
they blur the distinction we usually make between modifying or adding to a
program and simply using it.  Linking a program with a library, without
changing the library, is in some sense simply using the library, and is
analogous to running a utility program or application program.  However, in
a textual and legal sense, the linked executable is a combined work, a
derivative of the original library, and the ordinary General Public License
treats it as such.

  Because of this blurred distinction, using the ordinary General
Public License for libraries did not effectively promote software
sharing, because most developers did not use the libraries.  We
concluded that weaker conditions might promote sharing better.

  However, unrestricted linking of non-free programs would deprive the
users of those programs of all benefit from the free status of the
libraries themselves.  This Library General Public License is intended to
permit developers of non-free programs to use free libraries, while
preserving your freedom as a user of such programs to change the free
libraries that are incorporated in them.  (We have not seen how to achieve
this as regards changes in header files, but we have achieved it as regards
changes in the actual functions of the Library.)  The hope is that this
will lead to faster development of free libraries.

  The precise terms and conditions for copying, distribution and
modification follow.  Pay close attention to the difference between a
``work based on the library'' and a ``work that uses the library''.  The
former contains code derived from the library, while the latter only
works together with the library.

  Note that it is possible for a library to be covered by the ordinary
General Public License rather than by this special one.

\section*{Terms and Conditions for Copying, Distribution and Modification}

\begin{enumerate}
\setcounter{enumi}{-1}
\item This License Agreement applies to any software library which
contains a notice placed by the copyright holder or other authorized
party saying it may be distributed under the terms of this Library
General Public License (also called ``this License'').  Each licensee is
addressed as ``you''.

  A ``library'' means a collection of software functions and/or data
prepared so as to be conveniently linked with application programs
(which use some of those functions and data) to form executables.

  The ``Library'', below, refers to any such software library or work
which has been distributed under these terms.  A ``work based on the
Library'' means either the Library or any derivative work under
copyright law: that is to say, a work containing the Library or a
portion of it, either verbatim or with modifications and/or translated
straightforwardly into another language.  (Hereinafter, translation is
included without limitation in the term ``modification''.)

``Source code'' for a work means the preferred form of the work for
making modifications to it.  For a library, complete source code means
all the source code for all modules it contains, plus any associated
interface definition files, plus the scripts used to control compilation
and installation of the library.

  Activities other than copying, distribution and modification are not
covered by this License; they are outside its scope.  The act of
running a program using the Library is not restricted, and output from
such a program is covered only if its contents constitute a work based
on the Library (independent of the use of the Library in a tool for
writing it).  Whether that is true depends on what the Library does
and what the program that uses the Library does.
  
\item You may copy and distribute verbatim copies of the Library's
complete source code as you receive it, in any medium, provided that
you conspicuously and appropriately publish on each copy an
appropriate copyright notice and disclaimer of warranty; keep intact
all the notices that refer to this License and to the absence of any
warranty; and distribute a copy of this License along with the
Library.

  You may charge a fee for the physical act of transferring a copy,
and you may at your option offer warranty protection in exchange for a
fee.

\item You may modify your copy or copies of the Library or any portion
of it, thus forming a work based on the Library, and copy and
distribute such modifications or work under the terms of Section 1
above, provided that you also meet all of these conditions:

\begin{enumerate}
\item[a.]
The modified work must itself be a software library.

\item[b.]
You must cause the files modified to carry prominent notices
stating that you changed the files and the date of any change.

\item[c.]
You must cause the whole of the work to be licensed at no
charge to all third parties under the terms of this License.

\item[d.]
  If a facility in the modified Library refers to a function or a
  table of data to be supplied by an application program that uses the
  facility, other than as an argument passed when the facility is
  invoked, then you must make a good faith effort to ensure that, in
  the event an application does not supply such function or table, the
  facility still operates, and performs whatever part of its purpose
  remains meaningful.

    (For example, a function in a library to compute square roots has
    a purpose that is entirely well-defined independent of the
    application.  Therefore, Subsection 2d requires that any
    application-supplied function or table used by this function must
    be optional: if the application does not supply it, the square
    root function must still compute square roots.)
\end{enumerate}

These requirements apply to the modified work as a whole.  If
identifiable sections of that work are not derived from the Library,
and can be reasonably considered independent and separate works in
themselves, then this License, and its terms, do not apply to those
sections when you distribute them as separate works.  But when you
distribute the same sections as part of a whole which is a work based
on the Library, the distribution of the whole must be on the terms of
this License, whose permissions for other licensees extend to the
entire whole, and thus to each and every part regardless of who wrote
it.

Thus, it is not the intent of this section to claim rights or contest
your rights to work written entirely by you; rather, the intent is to
exercise the right to control the distribution of derivative or
collective works based on the Library.

In addition, mere aggregation of another work not based on the Library
with the Library (or with a work based on the Library) on a volume of
a storage or distribution medium does not bring the other work under
the scope of this License.

\item You may opt to apply the terms of the ordinary GNU General Public
License instead of this License to a given copy of the Library.  To do
this, you must alter all the notices that refer to this License, so
that they refer to the ordinary GNU General Public License, version 2,
instead of to this License.  (If a newer version than version 2 of the
ordinary GNU General Public License has appeared, then you can specify
that version instead if you wish.)  Do not make any other change in
these notices.

  Once this change is made in a given copy, it is irreversible for
that copy, so the ordinary GNU General Public License applies to all
subsequent copies and derivative works made from that copy.

  This option is useful when you wish to copy part of the code of
the Library into a program that is not a library.

\item You may copy and distribute the Library (or a portion or
derivative of it, under Section 2) in object code or executable form
under the terms of Sections 1 and 2 above provided that you accompany
it with the complete corresponding machine-readable source code, which
must be distributed under the terms of Sections 1 and 2 above on a
medium customarily used for software interchange.

  If distribution of object code is made by offering access to copy
from a designated place, then offering equivalent access to copy the
source code from the same place satisfies the requirement to
distribute the source code, even though third parties are not
compelled to copy the source along with the object code.

\item A program that contains no derivative of any portion of the
Library, but is designed to work with the Library by being compiled or
linked with it, is called a ``work that uses the Library''.  Such a
work, in isolation, is not a derivative work of the Library, and
therefore falls outside the scope of this License.

  However, linking a ``work that uses the Library'' with the Library
creates an executable that is a derivative of the Library (because it
contains portions of the Library), rather than a ``work that uses the
library''.  The executable is therefore covered by this License.
Section 6 states terms for distribution of such executables.

  When a ``work that uses the Library'' uses material from a header file
that is part of the Library, the object code for the work may be a
derivative work of the Library even though the source code is not.
Whether this is true is especially significant if the work can be
linked without the Library, or if the work is itself a library.  The
threshold for this to be true is not precisely defined by law.

  If such an object file uses only numerical parameters, data
structure layouts and accessors, and small macros and small inline
functions (ten lines or less in length), then the use of the object
file is unrestricted, regardless of whether it is legally a derivative
work.  (Executables containing this object code plus portions of the
Library will still fall under Section 6.)

  Otherwise, if the work is a derivative of the Library, you may
distribute the object code for the work under the terms of Section 6.
Any executables containing that work also fall under Section 6,
whether or not they are linked directly with the Library itself.

\item As an exception to the Sections above, you may also compile or
link a ``work that uses the Library'' with the Library to produce a
work containing portions of the Library, and distribute that work
under terms of your choice, provided that the terms permit
modification of the work for the customer's own use and reverse
engineering for debugging such modifications.

  You must give prominent notice with each copy of the work that the
Library is used in it and that the Library and its use are covered by
this License.  You must supply a copy of this License.  If the work
during execution displays copyright notices, you must include the
copyright notice for the Library among them, as well as a reference
directing the user to the copy of this License.  Also, you must do one
of these things:

\begin{enumerate}
\item[a.]  Accompany the work with the complete corresponding
  machine-readable source code for the Library including whatever
  changes were used in the work (which must be distributed under
  Sections 1 and 2 above); and, if the work is an executable linked
  with the Library, with the complete machine-readable "work that uses
  the Library", as object code and/or source code, so that the user
  can modify the Library and then relink to produce a modified
  executable containing the modified Library.  (It is understood that
  the user who changes the contents of definitions files in the
  Library will not necessarily be able to recompile the application to
  use the modified definitions.)

\item[b.] Accompany the work with a written offer, valid for at
    least three years, to give the same user the materials
    specified in Subsection 6a, above, for a charge no more
    than the cost of performing this distribution.

\item[c.] If distribution of the work is made by offering access to copy
    from a designated place, offer equivalent access to copy the above
    specified materials from the same place.

\item[d.] Verify that the user has already received a copy of these
    materials or that you have already sent this user a copy.
\end{enumerate}

  For an executable, the required form of the "work that uses the
Library" must include any data and utility programs needed for
reproducing the executable from it.  However, as a special exception,
the source code distributed need not include anything that is normally
distributed (in either source or binary form) with the major
components (compiler, kernel, and so on) of the operating system on
which the executable runs, unless that component itself accompanies
the executable.

  It may happen that this requirement contradicts the license
restrictions of other proprietary libraries that do not normally
accompany the operating system.  Such a contradiction means you cannot
use both them and the Library together in an executable that you
distribute.

\item You may place library facilities that are a work based on the
Library side-by-side in a single library together with other library
facilities not covered by this License, and distribute such a combined
library, provided that the separate distribution of the work based on
the Library and of the other library facilities is otherwise
permitted, and provided that you do these two things:

\begin{enumerate}
\item[a.] Accompany the combined library with a copy of the same work
    based on the Library, uncombined with any other library
    facilities.  This must be distributed under the terms of the
    Sections above.

\item[b.] Give prominent notice with the combined library of the fact
    that part of it is a work based on the Library, and explaining
    where to find the accompanying uncombined form of the same work.
\end{enumerate}

\item You may not copy, modify, sublicense, link with, or distribute
the Library except as expressly provided under this License.  Any
attempt otherwise to copy, modify, sublicense, link with, or
distribute the Library is void, and will automatically terminate your
rights under this License.  However, parties who have received copies,
or rights, from you under this License will not have their licenses
terminated so long as such parties remain in full compliance.

\item You are not required to accept this License, since you have not
signed it.  However, nothing else grants you permission to modify or
distribute the Library or its derivative works.  These actions are
prohibited by law if you do not accept this License.  Therefore, by
modifying or distributing the Library (or any work based on the
Library), you indicate your acceptance of this License to do so, and
all its terms and conditions for copying, distributing or modifying
the Library or works based on it.

\item Each time you redistribute the Library (or any work based on the
Library), the recipient automatically receives a license from the
original licensor to copy, distribute, link with or modify the Library
subject to these terms and conditions.  You may not impose any further
restrictions on the recipients' exercise of the rights granted herein.
You are not responsible for enforcing compliance by third parties to
this License.

\item If, as a consequence of a court judgment or allegation of patent
infringement or for any other reason (not limited to patent issues),
conditions are imposed on you (whether by court order, agreement or
otherwise) that contradict the conditions of this License, they do not
excuse you from the conditions of this License.  If you cannot
distribute so as to satisfy simultaneously your obligations under this
License and any other pertinent obligations, then as a consequence you
may not distribute the Library at all.  For example, if a patent
license would not permit royalty-free redistribution of the Library by
all those who receive copies directly or indirectly through you, then
the only way you could satisfy both it and this License would be to
refrain entirely from distribution of the Library.

If any portion of this section is held invalid or unenforceable under any
particular circumstance, the balance of the section is intended to apply,
and the section as a whole is intended to apply in other circumstances.

It is not the purpose of this section to induce you to infringe any
patents or other property right claims or to contest validity of any
such claims; this section has the sole purpose of protecting the
integrity of the free software distribution system which is
implemented by public license practices.  Many people have made
generous contributions to the wide range of software distributed
through that system in reliance on consistent application of that
system; it is up to the author/donor to decide if he or she is willing
to distribute software through any other system and a licensee cannot
impose that choice.

This section is intended to make thoroughly clear what is believed to
be a consequence of the rest of this License.

\item If the distribution and/or use of the Library is restricted in
certain countries either by patents or by copyrighted interfaces, the
original copyright holder who places the Library under this License may add
an explicit geographical distribution limitation excluding those countries,
so that distribution is permitted only in or among countries not thus
excluded.  In such case, this License incorporates the limitation as if
written in the body of this License.

\item The Free Software Foundation may publish revised and/or new
versions of the Library General Public License from time to time.
Such new versions will be similar in spirit to the present version,
but may differ in detail to address new problems or concerns.

Each version is given a distinguishing version number.  If the Library
specifies a version number of this License which applies to it and
"any later version", you have the option of following the terms and
conditions either of that version or of any later version published by
the Free Software Foundation.  If the Library does not specify a
license version number, you may choose any version ever published by
the Free Software Foundation.

\item If you wish to incorporate parts of the Library into other free
programs whose distribution conditions are incompatible with these,
write to the author to ask for permission.  For software which is
copyrighted by the Free Software Foundation, write to the Free
Software Foundation; we sometimes make exceptions for this.  Our
decision will be guided by the two goals of preserving the free status
of all derivatives of our free software and of promoting the sharing
and reuse of software generally.
\end{enumerate}

\section*{NO WARRANTY}

\begin{enumerate}
\setcounter{enumi}{10}
\item BECAUSE THE LIBRARY IS LICENSED FREE OF CHARGE, THERE IS NO
WARRANTY FOR THE LIBRARY, TO THE EXTENT PERMITTED BY APPLICABLE LAW\@.
EXCEPT WHEN OTHERWISE STATED IN WRITING THE COPYRIGHT HOLDERS AND/OR
OTHER PARTIES PROVIDE THE LIBRARY "AS IS" WITHOUT WARRANTY OF ANY
KIND, EITHER EXPRESSED OR IMPLIED, INCLUDING, BUT NOT LIMITED TO, THE
IMPLIED WARRANTIES OF MERCHANTABILITY AND FITNESS FOR A PARTICULAR
PURPOSE\@.  THE ENTIRE RISK AS TO THE QUALITY AND PERFORMANCE OF THE
LIBRARY IS WITH YOU\@.  SHOULD THE LIBRARY PROVE DEFECTIVE, YOU ASSUME
THE COST OF ALL NECESSARY SERVICING, REPAIR OR CORRECTION.

\item IN NO EVENT UNLESS REQUIRED BY APPLICABLE LAW OR AGREED TO IN
WRITING WILL ANY COPYRIGHT HOLDER, OR ANY OTHER PARTY WHO MAY MODIFY
AND/OR REDISTRIBUTE THE LIBRARY AS PERMITTED ABOVE, BE LIABLE TO YOU
FOR DAMAGES, INCLUDING ANY GENERAL, SPECIAL, INCIDENTAL OR
CONSEQUENTIAL DAMAGES ARISING OUT OF THE USE OR INABILITY TO USE THE
LIBRARY (INCLUDING BUT NOT LIMITED TO LOSS OF DATA OR DATA BEING
RENDERED INACCURATE OR LOSSES SUSTAINED BY YOU OR THIRD PARTIES OR A
FAILURE OF THE LIBRARY TO OPERATE WITH ANY OTHER SOFTWARE), EVEN IF
SUCH HOLDER OR OTHER PARTY HAS BEEN ADVISED OF THE POSSIBILITY OF SUCH
DAMAGES.
\end{enumerate}

\section*{How to Apply These Terms to Your New Libraries}

  If you develop a new library, and you want it to be of the greatest
possible use to the public, we recommend making it free software that
everyone can redistribute and change.  You can do so by permitting
redistribution under these terms (or, alternatively, under the terms of the
ordinary General Public License).

  To apply these terms, attach the following notices to the library.  It is
safest to attach them to the start of each source file to most effectively
convey the exclusion of warranty; and each file should have at least the
``copyright'' line and a pointer to where the full notice is found.

\begin{quote}
{\em one line to give the library's name and a brief idea of what it does.}\\
Copyright (C) {\em year} {\em name of author}\\

    This library is free software; you can redistribute it and/or
    modify it under the terms of the GNU Library General Public
    License as published by the Free Software Foundation; either
    version 2 of the License, or (at your option) any later version.

    This library is distributed in the hope that it will be useful,
    but WITHOUT ANY WARRANTY; without even the implied warranty of
    MERCHANTABILITY or FITNESS FOR A PARTICULAR PURPOSE\@.  See the GNU
    Library General Public License for more details.

    You should have received a copy of the GNU Library General Public
    License along with this library; if not, write to the Free
    Software Foundation, Inc., 675 Mass Ave, Cambridge, MA 02139, USA.
\end{quote}

Also add information on how to contact you by electronic and paper mail.

You should also get your employer (if you work as a programmer) or your
school, if any, to sign a ``copyright disclaimer'' for the library, if
necessary.  Here is a sample; alter the names:

\begin{quote}
  Yoyodyne, Inc., hereby disclaims all copyright interest in the
  library `Frob' (a library for tweaking knobs) written by James Random Hacker.

  {\em signature of Ty Coon}, 1 April 1990\\
  Ty Coon, President of Vice
\end{quote}

That's all there is to it!



\marginbibliographystyle{plain}
\marginbibliography{gaol}

\printindex[idx]

\end{document}

%%% Local Variables:
%%% mode: latex
%%% TeX-master: t
%%% End:
